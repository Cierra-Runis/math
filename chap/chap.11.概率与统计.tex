\chapter{概率与统计}

\section{常见概率分布、期望与方差}

\subsection{常见离散型随机变量的概率分布、期望与方差}

\subsubsection{二项分布}

\begin{itemize}
  \item 记法:$B(n, p)$
  \item 意义:记 $X$ 为 $n$ 次实验中出现次数
  \item 分布:$P(X = k) = C_n^k p^k (1-p)^{n - k}$
  \item 期望:$np$
  \item 方差:$np(1 - p)$
\end{itemize}

\subsubsection{0-1 分布}

\begin{itemize}
  \item 记法:$B(1, p)$
  \item 意义:略
  \item 分布:略
  \item 期望:略
  \item 方差:略
\end{itemize}

\subsubsection{超几何分布}

\begin{itemize}
  \item 记法:$H(n, M, N)$
  \item 意义:记 $X$ 为含 $M$ 特殊的总体 $N$ 中取 $n$ 次所取特殊体数
  \item 分布:$P(X = k) = \dfrac{C_M^k C_{N - M}^{n - k}}{C_N^n}$
  \item 期望:$\dfrac{nM}{N}$
  \item 方差:$\dfrac{nM}{N} \left( 1 - \dfrac{M}{N} \right) \dfrac{N - n}{N - 1}$
\end{itemize}

\subsubsection{负二项分布}

\begin{itemize}
  \item 记法:$Nb(r, p)$
  \item 意义:记 $X$ 为事件第 $r$ 次出现时的实验次数
  \item 分布:$P(X = k) = C_{k - 1}^{r - 1} p^r (1-p)^{k - r}$
  \item 期望:$\dfrac{r}{p}$
  \item 方差:$\dfrac{r(1 - p)}{p^2}$
\end{itemize}

\subsubsection{几何分布}

\begin{itemize}
  \item 记法:$Nb(1, p)$ 或 $Ge(p)$
  \item 意义:略
  \item 分布:略
  \item 期望:略
  \item 方差:略
\end{itemize}

\subsubsection{泊松分布}

\begin{itemize}
  \item 记法:$P(\lambda)$
  \item 意义:略
  \item 分布:$P(X = k) = \dfrac{\lambda^k}{k!} e^{-\lambda}$
  \item 期望:$\lambda$
  \item 方差:$\lambda$
\end{itemize}


\subsection{常见连续型随机变量的概率分布、期望与方差}

\begin{table}[H]
  \small
  \centering
  \begin{tabular}{cccccc}
    \hline
    名称   & 记法                 & 意义 & 概率密度 $f_X(x)$                                                                       & 期望                   & 方差                      \\
    \hline
    均匀分布 & $U[a, b]$          & 略  & $\dfrac{1}{b - a},\ x \in [a, b]$                                                   & $\dfrac{a + b}{2}$   & $\dfrac{(b - a)^2}{12}$ \\
    正态分布 & $N(\mu, \sigma^2)$ & 略  & $\dfrac{1}{\sqrt{2\pi\sigma^2}} \exp\left( -\dfrac{(x - \mu)^2}{2\sigma^2} \right)$ & $\mu$                & $\sigma^2$              \\
    指数分布 & $Exp(\lambda)$     & 略  & $\lambda e^{-\lambda x},\ x \geqslant 0$                                            & $\dfrac{1}{\lambda}$ & $\dfrac{1}{\lambda^2}$  \\
    \hline
  \end{tabular}
\end{table}

\section{二维随机变量}

% TODO: 二维随机变量

\section{切比雪夫不等式、大数定律、中心极限定理}

有关大数定律、中心极限定理,参看:

\begin{itemize}
  \item \href{https://www.bilibili.com/video/av745826370}{一个视频教你快速区分和记住大数定律以及中心极限定理}
  \item \href{https://www.bilibili.com/video/av612459403}{【官方双语】但是什么是中心极限定理?}
\end{itemize}

\subsection{切比雪夫不等式}

% TODO:

\subsection{辛钦大数定律与伯努利大数定律}

\subsubsection{辛钦大数定律} \label{subsubsec:khinchin-law-of-large-numbers}

若随机变量 $X_1, \cdots, X_n, \cdots$ 独立同分布,也记作 $\mathrm{i.i.d}$,且它们所遵循的分布的期望为 $\mu$,则有
\[
  \dfrac{1}{n} \sum_{i = 1}^n X_i \overset{P}{\longrightarrow} \mu
\]

我们举个生活中的例子来解释这个式子所表达的意思:

现在有道数学题让你求某个随机变量的期望,你发现这道题好像有问题,想看看答案给出的期望是不是正确的,你正好会使用 Python 进行 $n$ 次模拟实验,那么你自然会将 $n$ 次模拟出来的结果加起来除以 $n$,用算出的频率当作期望.

而辛钦大数定律就是这样做的正确性的根据.

证明如下:

我们将讨论随机变量所遵循的分布的方差 $\sigma^2$ 存在时的特例,因为 \textbf{辛钦大数定律本身不要求此方差存在}.

将 $\dfrac{1}{n} \sum_{i = 1}^n X_i$ 记作随机变量 $Y_n$,易得 $E(Y_n) = \mu$ 且 $D(Y_n) = \dfrac{\sigma^2}{n}$.

由切比雪夫不等式,有
\[
  P(|Y_n - \mu| < \varepsilon) \geqslant 1 - \dfrac{\sigma^2}{n\varepsilon^2}
\]

取极限有
\[
  \lim_{n \to +\infty} P(|Y_n - \mu| < \varepsilon) = 1
\]
故得证.

\subsubsection{伯努利大数定律}

若随机变量 $X_n \sim B(n, p),\ n \in \mathbb{N}^*$,则有
\[
  \dfrac{X_n}{n} \overset{P}{\longrightarrow} p
\]

据说伯努利大数定律就是辛钦大数定律的一个特例,但很多书都没有具体的推导过程.

这里参考
\href{https://www.bilibili.com/video/av745161573}{【概率论】大数定律与中心极限定理(2)从辛钦大数定律到伯努利大数定律}
给出证明:

假定随机变量 $x_1, \cdots, x_n, \cdots$ 独立同分布,遵循两点分布 $B(1, p)$,这也被称为伯努利试验,其期望为 $p$.

那么随机变量 $X_n \sim B(n, p)$ 实际上就是在考查这 $n$ 次实验中成功的次数,显然有
\[
  X_n = \sum_{i = 1}^{n} x_i
\]

可以注意到本节中的 $X_n$ 对应 \ref{subsubsec:khinchin-law-of-large-numbers} 一节中的 $Y_n$,$x_i$ 对应 $X_i$,故
\[
  \dfrac{X_n}{n} = \dfrac{1}{n} \sum_{i = 1}^n x_i \overset{P}{\longrightarrow} p
\]

\subsection{中心极限定理}

% TODO: