\chapter{前言}

毫不夸张的说, 数学是最重要的一门学科——无论是生活中简单的四则运算, 还是各个专业学科, 都离不开数学. 我们也一直在学数学, 从小时候父母掰着手指教我们一二三, 到小学四则运算, 初中……

随着学习内容的深入, 你可能发现越来越多的书籍开始变得“不讲人话”, 开篇便是概念、定义, 尽是些无聊无趣的空中楼阁, 纯纯的做题机器. 可能到了后期才发现, 明明可以以另一种完全不同的角度切入、解释这些, 甚至一切都开始变得明朗. 那既然如此, 为什么这些书籍还是如此死板呢?

本书就是这样一本有趣的书, 以一种更高的视角“俯视”我们已学的内容, 希望各位在此途中有所收获.

\section*{本书特色}

本书使用 \LaTeX 进行排版, \LaTeX\ 是一个文档准备系统 (Document Preparing System) , 它非常适用于生成高印刷质量的科技类和数学类文档. 它也能够生成所有其他种类的文档, 小到简单的信件, 大到完整的书籍. \LaTeX\ 使用 \TeX\ 作为它的排版引擎. \cite{lshort}

本书的中文排版遵守部分\href{https://w3c.github.io/clreq}{《Requirements for Chinese Text Layout 中文排版需求》}的建议, 在此列出本书排版所遵守的部分规定如下:

\begin{itemize}
  \item 全文出现的表标题置于表格上方, 图标题置于图下方.
  \item 全文将尽可能避免使用脚注进行注释, 避免使用括号.
  \item 注码将紧跟被注内容:若被注文字为完整句, 则注码放在句号后, 反之被注内容为注码前的最短小句.
  \item 中文正文使用 \href{https://github.com/adobe-fonts/source-han-serif}{思源宋体}, 西文正文使用 \href{https://en.wikipedia.org/wiki/Computer_Modern}{Computer Modern} 字体, 代码使用 \href{https://github.com/microsoft/cascadia-code}{Cascadia Code} 字体.
  \item 采用 \linespreadsize 倍行距.
  \item PDF 中所有可点击的链接都标为 \href{https://www.color-hex.com/color/c678dd}{紫色}, 特别地, 除了章节页等特殊页面外, 页眉中页号都指向目录, 以便快速翻页.
\end{itemize}

\section*{勘误与支持}

由于本人水平有限, 书中难免存在一些错误或不准确的地方, 恳请各位读者批评指正. 若各位读者在阅读过程中产生了疑问或发现错误, 欢迎在本书 GitHub 仓库的 \href{https://github.com/Cierra-Runis/math/issues}{Issue} 版块进行反馈.

\creator
