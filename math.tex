\documentclass[
  UTF8, openany, 10pt,
  a4paper, oneside, fontset=none
]{ctexbook}

\usepackage{color, extsizes, float, fancyhdr,
  fontspec, footmisc, geometry, graphicx, hyperref,
  lastpage, makeidx, tikz, wasysym, xcolor, xeCJK
}
\makeindex

%%% Infos
\title{{\Huge{\textbf{数学}}}}
\author{Cierra\_Runis}
\date{\today}

%%% Fonts
% English
\setmainfont{TimesNewRoman}[
  Path           = ./fonts/TimesNewRoman/,
  Extension      = .ttf,
  BoldFont       = *-Bold,
  ItalicFont     = *-Italic,
  BoldItalicFont = *-BoldItalic
]
% Code
\setmonofont{CascadiaCode}[
  Path           = ./fonts/CascadiaCode/,
  Extension      = .ttf
]
% Chinese
\setCJKmainfont{SourceHanSerif}[
  Path           = ./fonts/SourceHanSerif/,
  Extension      = .otf,
  BoldFont       = *-Bold,
  ItalicFont     = *,
  BoldItalicFont = *-Bold
]

%%% Links
\hypersetup{
  pdftitle       = {数学},
  pdfcreator     = {Cierra\_Runis},
  colorlinks     = true,
  linkcolor      = orange,
  urlcolor       = orange,
  citecolor      = orange,
}

% 章节格式
\ctexset{
  space          = true,
  section        = {
    name         = {第, 节},
    number       = {\chinese{section}}
  },
}

% 文档格式
\linespread{1.5}

% 页眉页脚格式
\pagestyle{fancy}
\fancyhf{}
\fancyhead[L]{\leftmark}
\fancyfoot[C]{\hyperlink{footlinkcontent}{\thepage}}

% 文档开端
\begin{document}

  \frontmatter
  \maketitle
  \chapter{前言}

毫不夸张的说,数学是最重要的一门学科——无论是生活中简单的四则运算,还是各个专业学科,都离不开数学。我们也一直在学数学,从小时候父母掰着手指教我们一二三,到小学四则运算,初中……

随着学习内容的深入,你可能发现越来越多的书籍开始变得“不讲人话”,开篇便是概念、定义,尽是些无聊无趣的空中楼阁,纯纯的做题机器。可能到了后期才发现,明明可以以另一种完全不同的角度切入、解释这些,甚至一切都开始变得明朗。那既然如此,为什么这些书籍还是如此死板呢?

本书就是这样一本有趣的书,/// TODO:

\section*{本书特色}

本书使用 \LaTeX 进行排版,\LaTeX\ 是一个文档准备系统(Document Preparing System),它非常适用于生成高印刷质量的科技类和数学类文档。它也能够生成所有其他种类的文档,小到简单的信件,大到完整的书籍。\LaTeX\ 使用 \TeX\ 作为它的排版引擎。\footnote{引自\href{https://github.com/CTeX-org/lshort-zh-cn/}{《一份(不太)简短的 \LaTeXe 介绍》}}

本书的中文排版遵守部分《Requirements for Chinese Text Layout 中文排版需求》\footnote{详见 \href{https://w3c.github.io/clreq}{官方网页}} 的建议,在此列出本书排版所遵守的部分规定如下:

\begin{itemize}
  \item 全文出现的图表的标号或说明,放在图表的下方
  \item 全文尽可能使用脚注进行注释,避免使用括号
  \item 注码紧跟被注内容:若被注文字为完整句,则注码放在句号后,反之被注内容为注码前的最短小句
  \item 中文正文使用 思源宋体 \footnote{详见 \href{https://github.com/adobe-fonts/source-han-serif}{官方 GitHub 仓库}} 字体,西文正文使用 Times New Roman 字体
  \item 代码使用 Cascadia Code \footnote{详见 \href{https://github.com/microsoft/cascadia-code}{官方 GitHub 仓库}} 字体
  \item 采用 1.5 倍行距
  \item 所有可点击的链接都标为 {\color{orange} 橙色} \footnote{色号为 \#FF8000}
  \item 特别地,除了章节页等特殊页面外,页脚中橙色页号都指向目录,以便快速翻页
\end{itemize}

\section*{勘误与支持}

由于本人水平有限,书中难免存在一些错误或不准确的地方,恳请各位读者批评指正。若各位读者在阅读过程中产生了疑问或发现错误,欢迎在本书 GitHub 仓库的 \href{https://github.com/Cierra-Runis/math/issues}{Issue} 版块进行反馈。

\section*{致谢}

/// TODO:


  \mainmatter
  \hypertarget{footlinkcontent}{}
  \tableofcontents
  \input{chapter/part.1.基础知识.chapter.1.整数.tex}

  \backmatter
  \input{chapter/epilogue.tex}

\end{document}
