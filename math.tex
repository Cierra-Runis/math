\documentclass[
  UTF8, openany, 12pt,
  a4paper, twoside, fontset=none
]{ctexbook}

\usepackage{amsmath, amssymb, amsfonts, array, color, enumitem, extsizes, esint,
  float, fancyhdr, fontspec, footmisc, geometry, ntheorem,
  graphicx, gbt7714, lastpage, makeidx, mathtools, tikz, xcolor, xeCJK
}
\usepackage{hyperref}
\usepackage[labelsep=quad]{caption}
\usetikzlibrary{intersections,calc,angles,quotes}

\makeindex
\bibliographystyle{gbt7714-numerical}
\graphicspath{{figures/}}

%%% Fonts
% Code
\setmonofont{CascadiaCode}[
  Path           = ./fonts/CascadiaCode/,
  Extension      = .ttf
]
% Chinese
\setCJKmainfont{SourceHanSerif}[
  Path           = ./fonts/SourceHanSerif/,
  Extension      = .otf,
  BoldFont       = *-Bold,
  ItalicFont     = *,
  BoldItalicFont = *-Bold
]

\geometry{a4paper,left=3.67cm,right=2.67cm,top=2.54cm,bottom=2.54cm,head=1.5cm,foot=1.75cm}

\definecolor{purple}{HTML}{C678DD}

\newcommand\omicron{o}
\everymath{\displaystyle}
\theorembodyfont{\normalfont}
{
  \newtheorem{theorem}{定理}        % 主要结论, 需证明
  \newtheorem{lemma}{引理}          % 辅助定理, 常用于证明定理的中间步骤
  \newtheorem{corollary}{推论}      % 从定理直接得出的结果
  \newtheorem{proposition}{命题}    % 一般性结论, 重要性略低于定理
  \newtheorem{conjecture}{猜想}     % 尚未被证明的假设, 处于探索阶段
  \newtheorem{definition}{定义}     % 引入新术语、新概念
  \newtheorem{problem}{练习}        % 练习题
  \newtheorem{answer}{答案}         % 练习题答案
  \newtheorem{example}{例}         % 在已经给出定义或定理后, 给出具体的例子来说明或应用该定义或定理
}
{
  \theoremstyle{nonumberplain}
  \newtheorem{question}{问}        % 提出问题或疑问, 通常用于引导思考或讨论
  \newtheorem{solution}{解}        % 通用的解答环境
  \newtheorem{proof}{证}           % 证明环境
}



% % 定理编号全局连续
% \theoremstyle{definition}
% \newtheorem{theorem}{定理}        % 按章编号,例如 1.1, 1.2
% \newtheorem{lemma}{引理}
% \newtheorem{corollary}{推论}
% \newtheorem{proposition}{命题}
% \newtheorem{conjecture}{猜想}
% \newtheorem{definition}{定义}
% \newtheorem{example}{例}

% \newtheoremstyle{no-dots}% 样式名
%   {0pt}   % 上方间距
%   {0pt}   % 下方间距
%   {\normalfont}  % 正文体
%   {}      % 缩进量
%   {\bfseries} % 头字体(加粗)
%   {}      % 定理头后的标点(留空)
%   {.5em}  % 定理头和正文之间的空隙
%   {\thmname{#1}} % 自定义标题格式(不加编号)

% % 练习、答案全局计数器
% \newcounter{problemcounter}
% \newcounter{answercounter}
% \newtheorem{problem}[problemcounter]{练习}
% \newtheorem{answer}[answercounter]{答案}
% \providecommand*{\problemcounterautorefname}{练习}
% \providecommand*{\answercounterautorefname}{答案}

% % 不需要编号
% \theoremstyle{no-dots}
% \newtheorem*{question}{问}
% \newtheorem*{solution}{解}
% \makeatletter
% \renewcommand{\proofname}{证}
% \renewenvironment{proof}[1][\bfseries{\proofname}]{%
%   \par\pushQED{\qed}%
%   \normalfont \topsep6\p@\@plus6\p@\relax
%   \trivlist
%   \item[\hskip\labelsep#1\@addpunct{\quad}]% <-- 这里清除默认句号
% }{%
%   \popQED\endtrivlist\@endpefalse
% }
% % \renewenvironment{solution}{%
% %   \par\textbf{解}\quad
% %   \ignorespaces
% % }{\par}
% \makeatother

%%% Links
\hypersetup{
  pdftitle       = {数学},
  pdfcreator     = {Cierra\_Runis},
  colorlinks     = true,
  linkcolor      = purple,
  urlcolor       = purple,
  citecolor      = purple,
}

\def\figureautorefname{图}
\def\tableautorefname{表}
\def\chapterautorefname~#1\null{第~#1~章\null}
\def\sectionautorefname~#1\null{#1~小节\null}
\def\subsectionautorefname~#1\null{#1~小节\null}
\def\subsubsectionautorefname~#1\null{#1~小节\null}

% Setup table and figure captions
\captionsetup[table]{labelfont=bf, textfont=bf}
\captionsetup[figure]{labelfont=bf, textfont=bf}
\DeclareCaptionFont{bf}{\small\bfseries}
\renewcommand{\thefigure}{\thechapter-\arabic{figure}}
\renewcommand{\thetable}{\thechapter-\arabic{table}}


\ctexset{
  chapter = {
    name = {第,章}, % 章节名称
    number = \arabic{chapter}, % 章节编号格式
  }
}
\xeCJKsetup{CJKmath=true}
\def\linespreadsize{1.35}
\linespread{\linespreadsize}

% 页眉页脚格式
\pagestyle{fancy}
\fancyhf{}
\fancyhead[C]{\small\leftmark}
\fancyhead[RO]{\hyperref[toc]{\small\thepage}}
\fancyhead[LE]{\hyperref[toc]{\small\thepage}}
\fancypagestyle{plain}{
  \fancyhf{}
  \renewcommand{\headrulewidth}{0pt} % 页眉线宽
  \renewcommand{\footrulewidth}{0pt} % 页脚线宽
}

% 文档开端
\begin{document}
\pagenumbering{Alph}
\pdfbookmark{标题页}{title}
\thispagestyle{empty}

\vspace*{\stretch{1}}
\noindent\begin{minipage}{\textwidth}
  \raggedleft
  {\huge \bfseries 数学}
  \noindent\rule[-1ex]{\textwidth}{5pt}\\[2.5ex]
  \hfill\emph{\Large 更高视角下的数学}
\end{minipage}

\vspace{\stretch{1}}
\noindent\rlap{%
  \begin{minipage}{\textwidth}
    \linespread{2}\selectfont\raggedleft
    {\bfseries 编著:} \creator \\
    {\bfseries 版本:} 版本号 \version, \latestdate
  \end{minipage}%
}

\vspace{\stretch{2}}

\newpage\thispagestyle{empty}
\begin{quote}\footnotesize
  Copyright \copyright{} {\the\year} Cierra Runis.

  本书采用 \href{https://creativecommons.org/licenses/by-nc-sa/4.0/deed.zh}{知识共享署名-非商业性使用-相同方式共享 4.0 国际许可协议} 进行许可.

  除非另有说明, 本文档中的所有内容均可自由使用, 但必须遵守上述许可协议.

  本书的内容及代码均可在 \href{https://github.com/Cierra-Runis/math}{GitHub 仓库} 中获取.
\end{quote}


\frontmatter
\pagenumbering{Roman}
\chapter{前言}

毫不夸张的说,数学是最重要的一门学科——无论是生活中简单的四则运算,还是各个专业学科,都离不开数学. 我们也一直在学数学,从小时候父母掰着手指教我们一二三,到小学四则运算,初中……

随着学习内容的深入,你可能发现越来越多的书籍开始变得“不讲人话”,开篇便是概念、定义,尽是些无聊无趣的空中楼阁,纯纯的做题机器. 可能到了后期才发现,明明可以以另一种完全不同的角度切入、解释这些,甚至一切都开始变得明朗. 那既然如此,为什么这些书籍还是如此死板呢?

本书就是这样一本有趣的书,以一种更高的视角“俯视”我们已学的内容,希望各位在此途中有所收获.

\section*{本书特色}

本书使用 \LaTeX 进行排版,\LaTeX\ 是一个文档准备系统(Document Preparing System),它非常适用于生成高印刷质量的科技类和数学类文档. 它也能够生成所有其他种类的文档,小到简单的信件,大到完整的书籍. \LaTeX\ 使用 \TeX\ 作为它的排版引擎. \cite{lshort}

本书的中文排版遵守部分\href{https://w3c.github.io/clreq}{《Requirements for Chinese Text Layout 中文排版需求》}的建议,在此列出本书排版所遵守的部分规定如下:

\begin{itemize}
  \item 全文出现的表标题置于表格上方,图标题置于图下方.
  \item 全文将尽可能避免使用脚注进行注释,避免使用括号.
  \item 注码将紧跟被注内容:若被注文字为完整句,则注码放在句号后,反之被注内容为注码前的最短小句.
  \item 中文正文使用 \href{https://github.com/adobe-fonts/source-han-serif}{思源宋体},西文正文使用 \href{https://en.wikipedia.org/wiki/Computer_Modern}{Computer Modern} 字体,代码使用 \href{https://github.com/microsoft/cascadia-code}{Cascadia Code} 字体.
  \item 采用 \linespreadsize 倍行距.
  \item PDF 中所有可点击的链接都标为 \href{https://www.color-hex.com/color/c678dd}{紫色},特别地,除了章节页等特殊页面外,页眉中页号都指向目录,以便快速翻页.
\end{itemize}

\section*{勘误与支持}

由于本人水平有限,书中难免存在一些错误或不准确的地方,恳请各位读者批评指正. 若各位读者在阅读过程中产生了疑问或发现错误,欢迎在本书 GitHub 仓库的 \href{https://github.com/Cierra-Runis/math/issues}{Issue} 版块进行反馈.

\begin{flushright}
  \href{https://github.com/Cierra-Runis}{\creator} \\
  \latestdate
\end{flushright}

\tableofcontents\label{toc}

\mainmatter
\chapter{自然数} \label{chap:natural-number}

自然数集 $\mathbb{N}$ 是我们接触最早的数集,但就是这样一个“性质简单”的数集,我们也能从中发现一些有趣的东西.

\section{认识自然数} \label{sec:understand-natural-number}

自我们出生就一直和数字打交道:人有几根手指?河里有几只鸭子?朴素的我们掰着手指记数:0、1、2、3……10,至此我们会使用 0 到 10 的数字来计数了.

好像有什么不对劲,0 是自然数吗?

不同的书籍、领域给出的答案是不一样的,就根据 \href{https://zh.wikipedia.org/wiki/\%E7\%9A\%AE\%E4\%BA\%9A\%E8\%AF\%BA\%E5\%85\%AC\%E7\%90\%86}{皮亚诺公理} 得出的结论,答案是肯定的.

这五条公理的内容如下:

\begin{enumerate}
  \item 0 是自然数.
  \item 每个自然数 $a$ 有且仅有一个后继数 $a'$ 为自然数.
  \item 对自然数 $b$、$c$,当且仅当 $b' = c'$ 时有 $b = c$.
  \item 0 不是任何自然数的后继数.
  \item 任意有关自然数的命题,若其对自然数 0 成立,且假定它对自然数 $a$ 为真时,可证其对 $a'$ 也成立,那么命题对所有自然数成立.
\end{enumerate}

个人认为把 0 认作自然数的好:离散数学中有单位元(幺元)的概念,若 $0 \notin \mathbb{N}$,则定义在集合 $\mathbb{N}$ 上的二元运算 $+$ 就没有单位元了,让其失去这样的性质就失去了数学的美感.

为了避免 $\mathbb{N}$ 的歧义,我们将尽可能使用 $\mathbb{N}_0$ 表示包含 0 的自然数集,而 $\mathbb{N}^*$ 则表示不包含 0 的自然数集. 当使用 $\mathbb{N}$ 时,默认是包含 0 的自然数集.

\section{四则运算} \label{sec:four-arithmetic-operations}

\subsection{加法} \label{subsec:addition}

\subsection{减法} \label{subsec:subtraction}

\begin{question}
  证明关于 $x_1, x_2, \cdots, x_n$ 的方程组
  \[
    \left\{
    \begin{gathered}
      \prod_{k = 1}^{n} (x_k - \lambda_1) = \mu_1 \\
      \prod_{k = 1}^{n} (x_k - \lambda_2) = \mu_2 \\
      \cdots                                      \\
      \prod_{k = 1}^{n} (x_k - \lambda_n) = \mu_n
    \end{gathered}
    \right.
  \]
  有且仅有确定的解.
\end{question}

\begin{solution}
  设 $f(x) = \prod_{k = 1}^{n} (x - \lambda_k)$,则 $f(x)$ 为 $n$ 次多项式,且其系数为 $\lambda_1, \lambda_2, \cdots, \lambda_n$ 的函数.

  由代数基本定理知 $f(x)$ 有 $n$ 个根,且这些根是 $\lambda_1, \lambda_2, \cdots, \lambda_n$ 的排列.

  故方程组有且仅有确定的解.
\end{solution}

\subsection{乘法} \label{subsec:multiplication}

\subsection{除法} \label{subsec:division}

\subsubsection{取余} \label{subsubsec:modulus}

\section{数的性质} \label{sec:number-properties}

\subsection{负数} \label{subsec:negative-number}

\subsection{奇数} \label{subsec:odd-number}

\subsection{偶数} \label{subsec:even-number}

\subsection{因数} \label{subsec:factor}

\subsection{倍数} \label{subsec:multiple}

\subsection{倒数} \label{subsec:reciprocal}

\subsection{质数} \label{subsec:prime-number}

\subsection{合数} \label{subsec:composite-number}

\section{比较大小} \label{sec:compare-size}

给出自然数 $a$、$b$,现在的我们看到不等式 $a > b$ 时,很容易就得到 $a - b > 0$,未免陷入到了代数的桎梏中——让我们回忆一下我们当初是怎么进行比较的:

\begin{figure}[H]
  \small
  \centering
  \begin{tikzpicture}
    \foreach \n/\t in
      {0/1,1/2,2/3,3/4}
      {\node[circle,fill=lightgray,draw]
        at (\n,1.2) {$\t$};}

    \foreach \n in
      {0,1,2}
      {\node at (\n,0.6) {|};}

    \foreach \n/\t in
      {0/1,1/2,2/3}
      {\node[circle,fill=lightgray,draw]
        at (\n,0) {$\t$};}
  \end{tikzpicture}
  \caption{比较大小} \label{fig:compare}
\end{figure}

像这样 \strong{一一对应},直到一方比另一方多或相等.

为什么在这讲如此简单的东西呢?我们以一个数学问题切入感受一下:

\begin{question}
  正奇数集 $O = \{1, 3, \cdots\}$ 与正偶数集 $E = \{2, 4, \cdots\}$ 哪个集合的元素个数(势)更大?
\end{question}

\begin{solution}
  存在函数 $f: O \to E, f(x) = x + 1$ 为双射函数,则 $O, E$ 同势.
\end{solution}

而双射的定义如下:

\begin{definition}[双射函数]
  若 $f: X \to Y$ 有 $\forall x \in X$ 存在唯一的 $y$ 与 $x$ 对应,且 $\forall y \in Y$ 存在唯一的 $x$ 与 $y$ 对应,则称 $f$ 为双射函数. 换句话说,如果 $f$ 为两集合间的 \strong{一一对应} 关系,则 $f$ 是双射的.
\end{definition}

在此也请读者们思考:

\begin{problem}[\cref{ans:odd-even-union}] \label{prob:odd-even-union}
$\mathbb{N}^*$ 与 $O, E$ 是否同势?若是请给出函数 $f$,否则给出理由.
\end{problem}

\begin{problem}[\cref{ans:real-number-interval}] \label{prob:real-number-interval}
集合 $A = \{x | 0 < x < 1, x \in \mathbb{R}\}$ 与集合 $\mathbb{R}$ 是否同势?若是请给出函数 $f$,否则给出理由.
\end{problem}

\section{进制} \label{sec:base}

\section{统计} \label{sec:statistics}

\subsection{平均数} \label{subsec:mean}

\subsection{最值} \label{subsec:max-min}

在一组数 $x_1, x_2, \cdots, x_n$ 中,若其中的某个数 $x_k$ 对其中任意一个数 $s$ 有 $x_k \geqslant s$,则称 $x_k$ 为最大值,同理可以给出最小值的定义.

那么根据这样的定义,最大值实际上是可以有“多个”的,比如一下一组数:$0, 1, 1, 2, 2$, 实际上后两个数都是这组数的最大值,只是两者相同而已.

为了避免这样的重复,我们一般将这组数用一个集合表示,而集合中的元素只会出现一次,那么我们搬出严格定义如下:

\begin{definition}[最大元与最小元] \label{def:max-min-element}
  设 $(P,\leqslant )$为偏序集,$S$ 为其子集. 若 $g \in S$ 对任何 $s \in S$ 有 $s \leqslant g$,则 $g$ 称为 $S$ 的 \strong{最大元}(Greatest Element),同理若 $l \in S$ 对任何 $s \in S$ 有 $l \leqslant s$,则 $l$ 称为 $S$ 的 \strong{最小元}(Least Element).
\end{definition}

我们会用多种方式来表示最大值:如 $s_{\max}$、$\max{S}$,在元素个数为二时,我们还可以用一个算式来表达最大值:

记 $M = \max{\{a, b\}}$,则
\begin{equation}
  M = \frac{a + b + |a - b|}{2} \label{eq:max}
\end{equation}

容易证明 $a \geqslant b$ 时 $M = a$,$a < b$ 时 $M = b$.

同理记 $m = \min{\{a, b\}}$,则
\begin{equation}
  m = \frac{a + b - |a - b|}{2} \label{eq:min}
\end{equation}

由此我们就将比较抽象的取最大最小值用准确的算式表达出来了,在未来我们学习随机变量时会利用这个算式的.

\section{坐标} \label{sec:coordinate}

\section{鸡兔同笼} \label{sec:chicken-rabbit}

\input{chap/chap.2.平面几何}
\input{chap/chap.3.集合与常用逻辑用语}
\input{chap/chap.4.一元二次函数、方程与不等式}
\input{chap/chap.5.函数}
\input{chap/chap.6.向量与几何}
\input{chap/chap.7.复数}
\input{chap/chap.8.直线与圆}
\input{chap/chap.9.圆锥曲线}
\input{chap/chap.10.数列与级数}

\backmatter
\chapter{参考答案}

\begin{answer}[\cref{prob:odd-even-union}] \label{ans:odd-even-union}
  \begin{solution}
    有 $f: \mathbb{N}^* \to O, f(x) = 2x - 1$ 为双射函数,则 $\mathbb{N}^*, O$ 同势.
    有 $f: \mathbb{N}^* \to E, f(x) = 2x$ 为双射函数,则 $\mathbb{N}^*, E$ 同势.
    实际上,同势具有传递性,因此 $\mathbb{N}^*, O, E$ 同势.
  \end{solution}
\end{answer}

\begin{answer}[\cref{prob:real-number-interval}] \label{ans:real-number-interval}
  \begin{solution}
    有 $f: \mathbb{R} \to (0, 1), f(x) = \dfrac{x}{1 + |x|}$ 为双射函数,则 $\mathbb{R}$ 与 $(0, 1)$ 同势.
  \end{solution}
\end{answer}

\input{chap/epilogue}
\renewcommand*{\bibnumfmt}[1]{\textbf{[#1]}}  % 加粗编号
\bibliographystyle{gbt7714-numerical}
\bibliography{reference}
\addcontentsline{toc}{chapter}{\bibname}

\printindex

\end{document}
