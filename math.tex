\documentclass[
  UTF8, openany, 12pt,
  a4paper, twoside, fontset=none
]{ctexbook}
\usepackage{amsmath, amssymb, amsfonts, array, color, enumitem, extsizes, esint,
  float, fancyhdr, fontspec, footmisc, geometry,
  graphicx, gbt7714, lastpage, makeidx, mathtools, tikz, xcolor, xeCJK
}
\usepackage[CJKbookmarks=true]{hyperref}
\usepackage{cleveref}
\usepackage[amsmath,thmmarks,hyperref]{ntheorem}
\usepackage[labelsep=quad]{caption}
\usetikzlibrary{intersections,calc,angles,quotes}

\def\creator{Cierra\_Runis}
\def\version{0.0.1}
\def\latestdate{{\the\year} 年 {\the\month} 月}

\definecolor{purple}{HTML}{C678DD}
\def\linespreadsize{1.35}

\makeindex
\graphicspath{{figures/}}

%%% Fonts
% Code
\setmonofont{CascadiaCode}[
  Path           = ./fonts/CascadiaCode/,
  Extension      = .ttf
]
% Chinese
\setCJKmainfont{SourceHanSerif}[
  Path           = ./fonts/SourceHanSerif/,
  Extension      = .otf,
  BoldFont       = *-Bold,
  ItalicFont     = *,
  BoldItalicFont = *-Bold
]

\geometry{a4paper,left=3.67cm,right=2.67cm,top=2.54cm,bottom=2.54cm,head=1.5cm,foot=1.75cm}


\ctexset{
  chapter = {
    name = {第,章}, % 章节名称
    number = \arabic{chapter}, % 章节编号格式
  }
}
\xeCJKsetup{CJKmath=true}

\crefformat{equation}{#2式~#1#3}
\crefformat{figure}{#2图~#1#3}
\crefformat{table}{#2表~#1#3}
\crefformat{chapter}{#2第~#1~章#3}
\crefformat{section}{#2第~#1~节#3}
\crefformat{subsection}{#2第~#1~节#3}
\crefformat{subsubsection}{#2第~#1~节#3}

\newcommand\omicron{o}
\everymath{\displaystyle}
\theorembodyfont{\normalfont}
{
  \newtheorem{theorem}{定理}        % 主要结论, 需证明
  \newtheorem{lemma}{引理}          % 辅助定理, 常用于证明定理的中间步骤
  \newtheorem{corollary}{推论}      % 从定理直接得出的结果
  \newtheorem{proposition}{命题}    % 一般性结论, 重要性略低于定理
  \newtheorem{conjecture}{猜想}     % 尚未被证明的假设, 处于探索阶段
  \newtheorem{definition}{定义}     % 引入新术语、新概念
  \newtheorem{problem}{练习}        % 练习题
  \newtheorem{answer}{答案}         % 练习题答案
}
{
  \theoremstyle{nonumberplain}
  \theoremseparator{\quad}
  \newtheorem{example}{例}         % 在已经给出定义或定理后, 给出具体的例子来说明或应用该定义或定理
  \newtheorem{question}{问}        % 提出问题或疑问, 通常用于引导思考或讨论
  \newtheorem{solution}{解}        % 通用的解答环境
}
{
  \theoremstyle{nonumberplain}
  \theoremsymbol{Q.E.D.}  % 在证明环境末尾加上一个证毕符号
  \newtheorem{proof}{证}           % 证明环境
}

\crefformat{theorem}{#2定理~#1#3}
\crefformat{lemma}{#2引理~#1#3}
\crefformat{corollary}{#2推论~#1#3}
\crefformat{proposition}{#2命题~#1#3}
\crefformat{conjecture}{#2猜想~#1#3}
\crefformat{definition}{#2定义~#1#3}
\crefformat{problem}{#2练习~#1#3}
\crefformat{answer}{#2答案~#1#3}

%%% Links
\hypersetup{
  pdftitle       = {数学},
  pdfcreator     = {Cierra\_Runis},
  colorlinks     = true,
  linkcolor      = purple,
  urlcolor       = purple,
  citecolor      = purple,
}

% Setup table and figure captions
\captionsetup[table]{labelfont=bf, textfont=bf}
\captionsetup[figure]{labelfont=bf, textfont=bf}
\DeclareCaptionFont{bf}{\small\bfseries}
\renewcommand{\thefigure}{\thechapter-\arabic{figure}}
\renewcommand{\thetable}{\thechapter-\arabic{table}}


\linespread{\linespreadsize}

% 页眉页脚格式
\pagestyle{fancy}
\fancyhf{}
\fancyhead[C]{\small\leftmark}
\fancyhead[RO]{\hyperref[toc]{\small\thepage}}
\fancyhead[LE]{\hyperref[toc]{\small\thepage}}
\fancypagestyle{plain}{
  \pagestyle{empty}
}

% 文档开端
\begin{document}
\pagenumbering{Alph}
\pdfbookmark{标题页}{title}
\thispagestyle{empty}

\vspace*{\stretch{1}}
\noindent\begin{minipage}{\textwidth}
  \raggedleft
  {\huge \bfseries 数学}
  \noindent\rule[-1ex]{\textwidth}{5pt}\\[2.5ex]
  \hfill\emph{\Large 更高视角下的数学}
\end{minipage}

\vspace{\stretch{1}}
\noindent\rlap{%
  \begin{minipage}{\textwidth}
    \linespread{2}\selectfont\raggedleft
    {\bfseries 编著:} Cierra Runis \\
    {\bfseries 版本:} 版本号 0.0.1, {\the\year} 年 {\the\month} 月
  \end{minipage}%
}

\vspace{\stretch{2}}

\newpage\thispagestyle{empty}
\begin{quote}\footnotesize
  Copyright \copyright{} {\the\year} Cierra Runis.

  本书采用 \href{https://creativecommons.org/licenses/by-nc-sa/4.0/deed.zh}{知识共享署名-非商业性使用-相同方式共享 4.0 国际许可协议} 进行许可.

  除非另有说明, 本文档中的所有内容均可自由使用, 但必须遵守上述许可协议.

  本书的内容及代码均可在 \href{https://github.com/Cierra-Runis/math}{GitHub 仓库} 中获取.
\end{quote}


\frontmatter
\pagenumbering{Roman}
\chapter{前言}

毫不夸张的说,数学是最重要的一门学科——无论是生活中简单的四则运算,还是各个专业学科,都离不开数学. 我们也一直在学数学,从小时候父母掰着手指教我们一二三,到小学四则运算,初中……

随着学习内容的深入,你可能发现越来越多的书籍开始变得“不讲人话”,开篇便是概念、定义,尽是些无聊无趣的空中楼阁,纯纯的做题机器. 可能到了后期才发现,明明可以以另一种完全不同的角度切入、解释这些,甚至一切都开始变得明朗. 那既然如此,为什么这些书籍还是如此死板呢?

本书就是这样一本有趣的书,以一种更高的视角“俯视”我们已学的内容,希望各位在此途中有所收获.

\section*{本书特色}

本书使用 \LaTeX 进行排版,\LaTeX\ 是一个文档准备系统(Document Preparing System),它非常适用于生成高印刷质量的科技类和数学类文档. 它也能够生成所有其他种类的文档,小到简单的信件,大到完整的书籍. \LaTeX\ 使用 \TeX\ 作为它的排版引擎. \cite{lshort}

本书的中文排版遵守部分\href{https://w3c.github.io/clreq}{《Requirements for Chinese Text Layout 中文排版需求》}的建议,在此列出本书排版所遵守的部分规定如下:

\begin{itemize}
  \item 全文出现的表标题置于表格上方,图标题置于图下方.
  \item 全文将尽可能避免使用脚注进行注释,同时避免使用括号.
  \item 注码将紧跟被注内容:若被注文字为完整句,则注码放在句号后,反之被注内容为注码前的最短小句.
  \item 中文正文使用 \href{https://github.com/adobe-fonts/source-han-serif}{思源宋体},西文正文使用 \href{https://en.wikipedia.org/wiki/Computer_Modern}{Computer Modern} 字体,代码使用 \href{https://github.com/microsoft/cascadia-code}{\texttt{Cascadia Code}} 字体.
  \item 采用 \linespreadsize 倍行距.
  \item PDF 中所有可点击的链接都标为 \href{https://www.color-hex.com/color/c678dd}{紫色}. 特别地,除了章节页等特殊页面外,页眉中页号都指向目录,以便快速翻页.
\end{itemize}

\section*{勘误与支持}

由于本人水平有限,书中难免存在一些错误或不准确的地方,恳请各位读者批评指正. 若各位读者在阅读过程中产生了疑问或发现错误,欢迎在本书 GitHub 仓库的 \href{https://github.com/Cierra-Runis/math/issues}{Issue} 版块进行反馈.

\begin{flushright}
  \href{https://github.com/Cierra-Runis}{\creator} \\
  \latestdate
\end{flushright}

\tableofcontents \label{toc}

\mainmatter
\chapter{自然数} \label{chap:natural-number}

自然数集 $\mathbb{N}$ 是我们接触最早的数集, 但就是这样一个“性质简单”的数集, 我们也能从中发现一些有趣的东西.

\section{认识自然数} \label{sec:understand-natural-number}

自我们出生就一直和数字打交道:人有几根手指?河里有几只鸭子?朴素的我们掰着手指记数:0、1、2、3……10, 至此我们会使用 0 到 10 的数字来计数了.

好像有什么不对劲, 0 是自然数吗?

不同的书籍、领域给出的答案是不一样的, 就根据 \href{https://zh.wikipedia.org/wiki/\%E7\%9A\%AE\%E4\%BA\%9A\%E8\%AF\%BA\%E5\%85\%AC\%E7\%90\%86}{皮亚诺公理} 得出的结论, 答案是肯定的.

这五条公理的内容如下:

\begin{enumerate}
  \item $0$ 是自然数.
  \item 每个自然数 $a$ 有且仅有一个后继数 $a'$ 为自然数.
  \item 对自然数 $b$、$c$, 当且仅当 $b' = c'$ 时有 $b = c$.
  \item $0$ 不是任何自然数的后继数.
  \item 任意有关自然数的命题, 若其对自然数 $0$ 成立, 且假定它对自然数 $a$ 为真时, 可证其对 $a'$ 也成立, 那么命题对所有自然数成立.
\end{enumerate}

个人认为把 0 认作自然数的好:离散数学中有单位元 (幺元) 的概念, 若 $0 \notin \mathbb{N}$, 则定义在集合 $\mathbb{N}$ 上的二元运算 $+$ 就没有单位元了, 让其失去这样的性质就失去了数学的美感.

\section{四则运算} \label{sec:four-arithmetic-operations}

\subsection{加法} \label{subsec:addition}

\subsection{减法} \label{subsec:subtraction}

\begin{question}
  证明关于 $x_1, x_2, \cdots, x_n$ 的方程组
  \[
    \left\{
    \begin{gathered}
      \prod_{k = 1}^{n} (x_k - \lambda_1) = \mu_1 \\
      \prod_{k = 1}^{n} (x_k - \lambda_2) = \mu_2 \\
      \cdots                                      \\
      \prod_{k = 1}^{n} (x_k - \lambda_n) = \mu_n
    \end{gathered}
    \right.
  \]
  有且仅有确定的解.
\end{question}

\begin{solution}
  设 $f(x) = \prod_{k = 1}^{n} (x - \lambda_k)$, 则 $f(x)$ 为 $n$ 次多项式, 且其系数为 $\lambda_1, \lambda_2, \cdots, \lambda_n$ 的函数.

  由代数基本定理知 $f(x)$ 有 $n$ 个根, 且这些根是 $\lambda_1, \lambda_2, \cdots, \lambda_n$ 的排列.

  故方程组有且仅有确定的解.
\end{solution}

\subsection{乘法} \label{subsec:multiplication}

\subsection{除法} \label{subsec:division}

\subsubsection{取余} \label{subsubsec:modulus}

\section{数的性质} \label{sec:number-properties}

\subsection{负数} \label{subsec:negative-number}

\subsection{奇数} \label{subsec:odd-number}

\subsection{偶数} \label{subsec:even-number}

\subsection{因数} \label{subsec:factor}

\subsection{倍数} \label{subsec:multiple}

\subsection{倒数} \label{subsec:reciprocal}

\subsection{质数} \label{subsec:prime-number}

\subsection{合数} \label{subsec:composite-number}

\section{比较大小} \label{sec:compare-size}

给出自然数 $a$、$b$, 现在的我们看到不等式 $a > b$ 时, 很容易就得到 $a - b > 0$, 未免陷入到了代数的桎梏中——让我们回忆一下我们当初是怎么进行比较的:

\begin{figure}[H]
  \small
  \centering
  \begin{tikzpicture}
    \foreach \n/\t in
      {0/1,1/2,2/3,3/4}
      {\node[circle,fill=lightgray,draw]
        at (\n,1.2) {$\t$};}

    \foreach \n in
      {0,1,2}
      {\node at (\n,0.6) {|};}

    \foreach \n/\t in
      {0/1,1/2,2/3}
      {\node[circle,fill=lightgray,draw]
        at (\n,0) {$\t$};}
  \end{tikzpicture}
  \caption{比较大小} \label{fig:compare}
\end{figure}

像这样 \strong{一一对应}, 直到一方比另一方多或相等.

为什么在这讲如此简单的东西呢?我们以一个数学问题切入感受一下:

\begin{question}
  正奇数集 $O = \{1, 3, \cdots\}$ 与正偶数集 $E = \{2, 4, \cdots\}$ 哪个集合的元素个数 (势) 更大?
\end{question}

\begin{solution}
  存在函数 $f: O \to E, f(x) = x + 1$ 为双射函数, 则 $O, E$ 同势
\end{solution}

而双射的定义如下:

\begin{definition}[双射函数]
  若 $f: X \to Y$ 有 $\forall x \in X$ 存在唯一的 $y$ 与 $x$ 对应, 且 $\forall y \in Y$ 存在唯一的 $x$ 与 $y$ 对应, 则称 $f$ 为双射函数. 换句话说, 如果 $f$ 为两集合间的 \strong{一一对应} 关系, 则 $f$ 是双射的.
\end{definition}

在此也请读者们思考:

\begin{problem}[\cref{ans:odd-even-union}] \label{prob:odd-even-union}
上文中的 $O \cup E$ 与 $O, E$ 是否同势?若是请给出函数 $f$, 否则给出理由.
\end{problem}

\begin{problem}[\cref{ans:real-number-interval}] \label{prob:real-number-interval}
集合 $A = \{x | 0 < x < 1, x \in \mathbb{R}\}$ 与集合 $\mathbb{R}$ 是否同势?若是请给出函数 $f$, 否则给出理由.
\end{problem}

\section{进制} \label{sec:base}

\section{统计} \label{sec:statistics}

\subsection{平均数} \label{subsec:mean}

\subsection{最值} \label{subsec:max-min}

在一组数 $x_1, x_2, \cdots, x_n$ 中, 若其中的某个数 $x_k$ 对其中任意一个数 $s$ 有 $x_k \geqslant s$, 则称 $x_k$ 为最大值, 同理可以给出最小值的定义.

那么根据这样的定义, 最大值实际上是可以有“多个”的, 比如一下一组数:$0, 1, 1, 2, 2$, 实际上后两个数都是这组数的最大值, 只是两者相同而已.

为了避免这样的重复, 我们一般将这组数用一个集合表示, 而集合中的元素只会出现一次, 那么我们搬出严格定义如下:

\begin{definition}[最大元与最小元] \label{def:max-min-element}
  设 $(P,\leqslant )$为偏序集, $S$ 为其子集. 若 $g \in S$ 对任何 $s \in S$ 有 $s \leqslant g$, 则 $g$ 称为 $S$ 的 \strong{最大元} (Greatest Element) , 同理若 $l \in S$ 对任何 $s \in S$ 有 $l \leqslant s$, 则 $l$ 称为 $S$ 的 \strong{最小元} (Least Element) .
\end{definition}

我们会用多种方式来表示最大值:如 $s_{\max}$、$\max{S}$, 在元素个数为二时, 我们还可以用一个算式来表达最大值:

记 $M = \max{\{a, b\}}$, 则
\begin{equation}
  M = \frac{a + b + |a - b|}{2} \label{eq:max}
\end{equation}

容易证明 $a \geqslant b$ 时 $M = a$, $a < b$ 时 $M = b$.

同理记 $m = \min{\{a, b\}}$, 则
\begin{equation}
  m = \frac{a + b - |a - b|}{2} \label{eq:min}
\end{equation}

由此我们就将比较抽象的取最大最小值用准确的算式表达出来了, 在未来我们学习随机变量时会利用这个算式的.


\section{坐标} \label{sec:coordinate}

\section{鸡兔同笼} \label{sec:chicken-rabbit}

% 4.认识物体和平面图形:长方体、正方体、圆柱和球等立体图形与长方形、正方形、三角形和圆等平面图形.

% 5.分类:单一标准的分类和不同标准的分类

% 6.6~9的认识和加减法: (1) 6、7的认识和加减法 (数数、数序、比大小、序数、写数、组成) .  (2) 8、9的认识和加减法 (出现了“一图两式”和“一图四式”、渗透统计思想、比多比少内容)  (3) 10的认识和有关10的加减法 (省略了10的序数意义、填未知加数) .  (4) 连加、连减和加减混合计算.  (5) 整理和复习.

% 7.11~20各数的认识:数数、读数、数序和大小、序数、写数、个位和十位、10加几和十几加减几 (不退位) 、十几减十.

% 8.认识钟表:认识钟面、认识整时、认识半时.

% 9.20以内的进位加法:9加几 (“点数”、“接着数”、“凑十”和“根据具体题目选择特殊方法”) , 8、7、6加几 (“拆小数, 凑十数”、“拆大数, 凑小数”和“交换加数的位置”) , 5、4、3、2加几和“用数学”.

% 一年级 (下)

% 1.位置:用“上、下, 前、后, 左、右”描述物体的相对位置. 根据行、列确定物体的位置.

% 2.20以内的退位减法:十几减9. 十几减几. 用数学.

% 3.图形的拼组:平面图形的特征. 立体图形的关系

% 4.100以内数的认识 :数的认识 (它包括:数数、数的组成、数位的含义、数的顺序) 和加减 (大小比较、整十数加一位数和相应的减法) .

% 5.认识人民币:认识人民币的单位元、角、分, 知道1元=10角, 1角=10分. 简单的计算.

% 6.100以内的加法和减法 (一) :口算整十数加、减整十数. 口算两位数加、减一位和整十数. 用加法和减法解决简单的问题.

% 7.认识时间:认识几时几分 (5分5分数、1分1分数) .

% 8.找规律:最简单的图形变化规律. 稍复杂的图形变化规律. 图形与数字变化规律. 数字变化规律.

% 9.统计:简单的条形统计图 (1格代表1) . 统计表. 提出问题.

% 二年级 (上)

% 1.长度单位:统一长度单位. 认识厘米 用厘米量. 认识米 用米量. 认识线段, 量、画线段.

% 2.100以内的加法和减法 (二) :两位数加两位数的不进位加法和进位加法. 两位数减两位数的不退位减法、退位减法. 连加、连减、加减混合运算和加减法估算.

% 3.角的初步认识:角和直角的初步认识.

% 4.表内乘法 (一) :乘法的初步认识. 2~6的乘法口诀及乘加、乘减两步计算式题. 用数学.

% 5.观察物体:从不同位置观察物体. 轴对称. 镜面对称.

% 6.表内乘法 (二) :7~9的乘法口诀. 用7~9的乘法口诀解决简单的实际问题 (求一个数的几倍是多少) .

% 7.统计:条形统计图 (1格表示2个) .

% 8.数学广角:简单的排列 (2个数字摆两位数) . 最简单的推理 (2个条件) . 简单推理 (3个条件) .

% 二年级 (下)

% 1.数学问题:运用加法和减法两步计算解决问题, 并学会使用小括号. 运用乘法和加法 (或减法) 两步计算解决问题.

% 2.表内除法 (一) :除法的初步认识. 用2~6的乘法口诀求商 (被除数不超过12, 被除数不超过36. ) . 解决问题 (用除法计算解决简单. 用乘法和除法两步计算解决简单实际问题的内容) .

% 3.图形与变换:认识锐角和钝角. 平移和旋转

% 4.表内除法 (二) :用7、8、9的乘法口诀求商. 解决问题 (求一个数是另一个数的几倍是多少. 用乘、除法两步计算的稍复杂的实际问题) .

% 5.万以内数的认识:千以内数的认识 (数数、读写、数的组成、大小比较) . 万以内数的认识 (认识、读写、组成、数位顺序、中间有0的数的读写、比大小、近似数) . 口算整百整千数的加减.

% 6.克和千克:建立1克和1千克的观念、克和千克之间的进率、认识常见的秤、解决问题.

% 7.万以内的加法和减法 (一) :口算两位数加、减两位数 (和在100以内) , 笔算几百几十加、减几百几十. 加、减法估算.

% 8.统计:认识条形统计图 (一格代表五个单位) 和简单的复式统计表、合理预测.

% 9.找规律:图形的变化规律. 数列的变化规律.

% 三年级 (上)

% 1.测量:毫米、分米的认识、千米的认识和吨的认识.

% 2.万以内的加法和减法 (二) :加法、减法 (两位数加两位数和是三位数的连续进位加法. 三位数加、减三位数中连续进位加和连续退位减) . 加、减法的验算.

% 3.四边形:四边形和平行四边形的初步认识, 周长的含义, 长方形和正方形周长计算公式的探索和应用, 对实物的估量等.

% 4.有余数的除法:意义和计算 (表内除法竖式的含义. 有余数除法竖式及余数的含义. 余数和除数的关系) . 解决问题.

% 5.时、分、秒:时间单位“秒”, 以及有关时间的简单计算.

% 6.多位数乘一位数:口算乘法 (整十、整百数乘一位数的口算和相应的估算) . 笔算乘法 (

% 7.分数的初步认识:几分之一、几分之几的认识, 简单的分数加、减法 (同分母加减、1减几分之几) .

% 8.可能性:事件发生的确定性和不确定性. 可能发生的结果. 事件发生的可能性有大小.

% 9.数学广角:两上衣和三裤子搭配的组合数. 3个数字摆三位数的排列. 4个队一共要踢多少场球的组合.

% 三年级 (下)

% 1.位置与方向:辨认东、南、西、北、东北、西北、东南和西南八个方向, 并认识简单的路线图.

% 2.除数是一位数的除法:口算除法 (用一位数除商是整十、整百、整千的数、用一位数除几百几十或几千几百、除法估算) . 笔算除法. 除法的验算. 有关0的除法.

% 3.统计:简单的数据分析 (横向条形统计图, 起始格与其他格代表的单位量不一致) . 求平均数.

% 4.年、月、日:认识年月日. 知道平年闰年. 知道24时计时法. 计算经过时间.

% 5.两位数乘两位数:口算. 估算. 笔算.

% 6.面积:面积和面积单位, 长方形、正方形的面积计算, 面积单位的进率, 常用的土地面积单位.

% 7.小数的初步认识:认识小数 (含义、读写法、大小比较) . 简单的小数加减法 (一位小数的加减法) .

% 8.解决问题:乘法 (或除法) 、乘法和除法两步. 乘加 (减) 、除减 (加) 两步计算解决问题.

% 9.数学广角:集合和等量代换.

% 四年级 (上)

% 1.大数的认识:亿以内数的认识. 亿以上数的认识. 计算工具的认识及用计算器计算.

% 2.角的度量:认识射线和直线, 知道线段、射线和直线的区别. 认识常见的几种角, 会比较角的大小, 会用量角器量角的度数和按指定度数画角.

% 3.三位数乘两位数:口算乘法, 笔算乘法, 常见数量关系—速度、时间和路程, 以及乘法的估算.

% 4.平行四边形和梯形:垂直与平行. 平行四边形和梯形的认识.

% 5.除数是两位数的除法:口算除法、笔算除法.

% 6.统计:横、纵向复式条形统计图.

% 7.数学广角:合理安排 (烙饼、沏茶、安排炒菜的顺序、码头卸货、田忌赛马)

% 四年级 (下)

% 1.四则运算:解决问题 (加减混合、乘除混合、两个商积之和差混合运算) . 三步式题. 引导总结.

% 2.确定位置:根据方向、距离确定物体的位置. 绘出物体的. 位置关系的相对性. 描述并绘制简单的线路图.

% 3.运算定律与简便计算:加法、乘法的交换律与结合律, 乘法对于加法的分配律, 以及这五条运算定律的一些比较简单的运用.

% 4.小数的意义和性质:小数的意义, 认识小数的计数单位, 会读、写小数, 会比较小数的大小. 小数的性质和小数点位置移动引起小数大小变化的规律. 小数和十进复名数的相互改写. 用“四舍五入法”保留一定的小数数位, 求出小数的近似数, 并能把较大的数改写成用万或亿作单位的小数.

% 5.三角形:三角形的特性、三角形两边之和大于第三边、三角形的分类、三角形内角和是180°及图形的拼组.

% 6.小数的加法和减法:小数加、减法、混合运算以及整数的运算定律推广到小数.

% 7.统计:折线统计图、

% 8.数学广角:植树问题

% 五年级 (上)

% 1. 小数乘法:小数乘法. 积的近似值. 连乘、乘加、乘减两步计算. 整数乘法运算定律推广到小数.

% 2. 小数除法:小数除以整数、一个数除以小数、商的近似值、循环小数、用计算器探索规律、解决问题 (连除、去尾法、归一法) .

% 3. 观察物体:从不同的位置观察物体, 所看到的形状是不同的. 使学生能正确辨认从正面、侧面和上面观察到的简单物体或两个及一组立体图形的位置关系和形状.

% 4. 简易方程:用字母表示数和解简易方程, 以及简易方程在解决一些实际问题中的运用.

% 5.多边形的面积:平行四边形的面积、三角形的面积、梯形的面积和组合图形的面积.

% 6. 统计与可能性:事件发生的等可能性以及游戏规则的公平性, 会求简单事件发生的概率. 理解中位数的意义, 会求数据的中位数.

% 7. 数学广角:数字编码.

% 五年级 (下)

% 1. 图形的变换:进一步认识图形的轴对称, 探索图形成轴对称的特征和性质, 学习在方格纸上画出一个图形的轴对称图形和画出一个简单图形旋转90°后的图形,发展空间观念.

% 2. 因数与倍数:因数、倍数. 2、5、3的倍数的特征. 质数、合数.

% 3. 长方体和正方体:长方体和正方体的认识, 长方体和正方体的表面积, 长方体和正方体的体积 (容积) .

% 4. 分数的意义和性质:分数的意义、分数与除法的关系, 真分数与假分数, 分数的基本性质, 最大公因数与约分, 最小公倍数与通分以及分数与小数的互化.

% 5.分数的加法和减法:分数加、减法的意义, 同分母分数加减法, 异分母分数加减法, 分数加减混合运算以及整数加法的运算定律推广到分数.

% 6. 统计:认识众数. 复式折线统计图.

% 7. 数学广角:找次品.

% 六年级 (上)

% 1. 位置:用数对确定位置.

% 2. 分数乘法:分数乘法、解决问题 (求一个数的几分之几是多少) 和倒数.

% 3. 分数除法:分数除法的意义与计算. 解决问题. 比的意义与基本性质, 求比值与化简比, 及其比的应用.

% 4. 圆:认识圆、圆的周长和圆的面积等.

% 5.百分数:百分数的意义和写法, 百分数和分数、小数的互化以及用百分数解决问题.

% 6. 统计:体会扇形统计图的特点和作用.

% 7. 数学广角:“鸡兔同笼”问题

% 六年级 (下)

% 1. 负数:初步认识负数, 能正确的读、写正数和负数, 知道0既不是正数也不是负数. 用负数表示一些日常生活中的实际问题. 比较正数、0和负数之间的大小.

% 2. 圆柱与圆锥:圆柱和圆锥的认识, 圆柱的表面积, 圆柱的体积和圆锥的体积.

% 3. 比例:比例的意义和基本性质. 成正比例和反比例的意义. 比例的应用.

% 4. 统计:综合应用学过的统计知识, 能从统计图中准确提取统计信息, 能够正确解释统计结果. 根据统计图提供的信息, 作出正确的判断或简单预测.

% 5. 数学广角:抽屉原理

% 6. 整理和复习:数与代数. 空间与图形. 统计与概率. 综合应用.
��\chapter{s^b��QUO}



\section{s^b��Vb_}



\section{�zSO�Vb_}



\section{�[�y}



\section{҉}


��\chapter{ƖTN8^(u;���(u�}


��\chapter{NCQ�N!k�Qpe0�ezN
NI{_}


��\chapter{�Qpe}



\section{�Qpe0�[pe0�yR}



\subsection{�W,g�wƋ}



\subsubsection{Y6��Qpe�v���l}



�[ $f(x)$��� $f'(x)$ :NN6��[�$f''(x)$ :N�N6��[�$f'''(x)$ :N	N6��[��Na6��[���S:N $f^{(n)}(x)$.



\subsubsection{�g<P�p$R+R�lR}



�[�Qpe $f(x)$ (W $x = x_0$ Y�X[(W $n \in \mathbb{N}^*$ O�_

\[

  f'(x_0) = f''(x_0) = \dots = f^{(n - 1)}(x_0) = 0, f^{(n)}(x_0) > 0

\]

RS_ $n$ :NvPpe�e�$x_0$ :N�g\<P�p.



O�_

\[

  f'(x_0) = f''(x_0) = \dots = f^{(n - 1)}(x_0) = 0, f^{(n)}(x_0) < 0

\]

RS_ $n$ :NvPpe�e�$x_0$ :N�g'Y<P�p. S_ $n$ :NGYpe�e $x_0$ �e
N/f�g'Y<P�p�_N
N/f�g\<P�p.



N,��S $n = 2$�sSN,�eg�

\begin{center}

  $f'(x_0) = 0, f''(x_0) > 0$ sS	g $x_0$ :N�g\<P�p. \\

  $f'(x_0) = 0, f''(x_0) < 0$ sS	g $x_0$ :N�g'Y<P�p.

\end{center}



% TODO: �la 18 t^hQ�V	NwS�[pe�� (2) �



\subsection{4tu
NI{_}



\begin{gather}

  \forall x \in D, f''(x) \ge0 \iff \forall x_1, x_2 \in D, \dfrac{f(x_1) + f(x_2)}{2} \ge f(\dfrac{x_1 + x_2}{2}) \label{eq:convex-inequality} \\

  \forall x \in D, f''(x) \le0 \iff \forall x_1, x_2 \in D, \dfrac{f(x_1) + f(x_2)}{2} \le f(\dfrac{x_1 + x_2}{2}) \label{eq:concave-inequality}

\end{gather}



ُ(W�V�P
N>f6qb�z.



\subsection{ALG 
NI{_}



\begin{equation}

  \forall x_1, x_2 \in \mathbb{R}^* \implies \dfrac{x_1 + x_2}{2} > \dfrac{x_1 - x_2}{\ln{x_1} - \ln{x_2}} > \sqrt{x_1x_2} \label{eq:alg-inequality}

\end{equation}



\subsection{	N6��[peN���p-N�p�[pesQ�|}



� $f'''(x) > 0$ N $f(x)$ 	g$N���p $x_1, x_2$�R

\begin{equation}

  f'\left(\dfrac{x_1 + x_2}{2}\right) < 0 \label{eq:deriv_middle_negative}

\end{equation}



� $f'''(x) < 0$ N $f(x)$ 	g$N���p $x_1, x_2$�R

\begin{equation}

  f'\left(\dfrac{x_1 + x_2}{2}\right) > 0 \label{eq:deriv_middle_positive}

\end{equation}



�sO(uZ����Qpe�l��f \cref{eq:deriv_middle_negative}�



\begin{proof}

  �� $g(x) = f(x) - f(a - x)$�vQ-N $a = x_1 + x_2$�N�N $x_1 < \dfrac{a}{2} < x_2$�R

  \begin{align*}

    g'(x)   & = f'(x) + f'(a - x)         \\

    g''(x)  & = f''(x) - f''(a - x)       \\

    g'''(x) & = f'''(x) + f'''(a - x) > 0

  \end{align*}

  1u

  \[

    g''(\dfrac{a}{2}) = 0

  \]

  �_

  \[

    g'(x) \ge g'(\dfrac{a}{2}) = 2f'(\dfrac{a}{2})

  \]

  GP��

  \[

    f'(\dfrac{a}{2}) \ge 0

  \]

  R

  \[

    g'(x) \ge 0

  \]

  sS�_ $g(x)$ US���X�FO

  \[

    g(\dfrac{a}{2}) = 0 \implies f(x_1) < f(x_2)

  \]

  N

  \[

    f(x_1) = f(x_2) = 0

  \]

  �v�����v�`�Ee

  \[

    f'(\dfrac{a}{2}) < 0

  \]

  sS

  \[

    f'(\dfrac{x_1 + x_2}{2}) < 0

  \]

\end{proof}



\subsection{)R(u���QpeBl�[Bl�f�~
NN�pR�~�e�s} \label{subsec:implicit-derivative}



	g $F(x,y) = x^2 + 2x + y^2 = 0$�R
NN�p $(x,y)$ R�~�e�s

\[

  k = \dfrac{\mathrm{d}y}{\mathrm{d}x} = -\dfrac{\dfrac{\partial F}{\partial x}}{\dfrac{\partial F}{\partial y}} = -\dfrac{2x + 2}{2y}= -\dfrac{x + 1}{y}

\]



vQ-N $\dfrac{\partial F}{\partial x}$ h�:y�[ $F$ Bl $x$ �vOP�[�sS\d��N $x$ �vvQ�N�SϑƉ:N*g�w8^peۏL�Bl�[.



Tt $\dfrac{\partial F}{\partial y}$ h�:y�[ $F$ Bl $y$ �vOP�[.



\subsection{)R(u���QpeBl�[Bl�~_gag�NN�v�Qpeg<P}



�]�w�~_gag�N $F(x,y) = 0$�Bl $G(x,y)$ g<P.



\begin{example}

  $F(x,y) = x^2 + y^2 + xy - 4 = 0$�Bl $G(x,y) = x^2 + y^2$ g<P.

\end{example}



\begin{solution}

  �N

  \[

    -\dfrac{\dfrac{\partial F}{\partial x}}{\dfrac{\partial F}{\partial y}} = -\dfrac{\dfrac{\partial G}{\partial x}}{\dfrac{\partial G}{\partial y}}

  \]

  S�{�_ $y = \pm x$.



  N $F(x,y) = 0$ T��z㉗_

  \[

    \left(\pm\dfrac{2}{\sqrt{3}}, \pm\dfrac{2}{\sqrt{3}}\right), (\pm2, \mp2)

  \]

  �NeQ�_ $G(x,y)$ g'Y<P:N $8$�g\<P:N $\dfrac{8}{3}$.

\end{solution}




N�[hQ%N(��FOnx�[�_Y�eP	g(u.



\subsection{W%��f�~
NN�pR�~�ez}



\begin{table}[H]

  \small

  \centering

  \caption{W%��f�~�ezNR�~�ez} \label{tab:conic-tangent}

  \begin{tabular}{cc}

    \hline

    �ez                                      & Ǐ�p $P(x_0, y_0)$ R�~�ez                       \\

    \hline

    $(x - a)^2 + (y - b)^2 = R^2$           & $(x - a)(x_0 - a) + (y - b)(y_0 - b) = R^2$ \\

    $\frac{x^2}{a^2} + \frac{y^2}{b^2} = 1$ & $\frac{xx_0}{a^2} + \frac{yy_0}{b^2} = 1$   \\

    $\frac{x^2}{a^2} - \frac{y^2}{b^2} = 1$ & $\frac{xx_0}{a^2} - \frac{yy_0}{b^2} = 1$   \\

    $\frac{y^2}{b^2} - \frac{x^2}{a^2} = 1$ & $\frac{yy_0}{b^2} - \frac{xx_0}{a^2} = 1$   \\

    $y^2 = 2px$                             & $yy_0 = p(x + x_0)$                         \\

    $x^2 = 2py$                             & $xx_0 = p(y + y_0)$                         \\

    \hline

  \end{tabular}

\end{table}



\subsection{�g<P�pOP�y}



\begin{example}

  $f(x) = x(\ln{x} - \dfrac{x}{2} + a - 1)$ 	g$N�g<P�p $x_1, x_2, x_1 < x_2$.

  \begin{enumerate}[leftmargin=*, label=(\arabic*)]

    \item Bl $a$ ��V.

    \item ��f $2 \ln{x_1} + \ln{x_2} < 0$.

  \end{enumerate}

\end{example}



\begin{enumerate}[leftmargin=*, label=(\arabic*)]

  \item \begin{solution}

          1u

          \[

            f'(x) = \ln{x} - x + a

          \]

          \[

            f''(x) = \dfrac{1}{x} - 1

          \]

          �_

          \[

            f'(x)_{\max} = f'(1) = a - 1

          \]

          1u�� $f'(x)_{\max} > 0$�Ee $a \in (1, +\infty)$.

        \end{solution}

  \item \begin{proof}

          1u (1) �S�_ $0 < x_1 < 1 < x_2$�@b���S_�S�N9e�� $x_2 < \dfrac{1}{x_1^2}$.



          �V:N

          \[

            \dfrac{1}{x_1^2}, x_2 \in (1, +\infty)

          \]

          N1u (1) 	g $f'(x)$ (W $(1, +\infty)$ 
NUS���Q�R@b��

          \[

            x_2 < \dfrac{1}{x_1^2}

          \]

          �S�N9e��

          \[

            f'(\dfrac{1}{x_1^2}) < f'(x_2) = 0

          \]

          �

          \[

            f'(\dfrac{1}{x_1^2}) = \ln{\dfrac{1}{x_1^2}} - \dfrac{1}{x_1^2} + a

          \]

          vQ-N�v $a$ �S�S1u

          \[

            f'(x_1) = \ln{x_1} - x_1 + a = 0

          \]

          �_0R�R�NeQ�_

          \[

            f'(\dfrac{1}{x_1^2}) = \ln{\dfrac{1}{x_1^2}} - \dfrac{1}{x_1^2} + x_1 - \ln{x_1}

          \]

          ��

          \[

            g(x) = \ln{\dfrac{1}{x^2}} - \dfrac{1}{x^2} + x - \ln{x} = -3 \ln{x} + x - \dfrac{1}{x^2}, x \in (0, 1)

          \]

          Ee��f $g(x) < 0$, f�_

          \[

            g'(x) = \dfrac{(x - 1)(x^2 - 2x - 2)}{x^3} > 0

          \]

          (W $(0, 1)$ 
Nb�z�R $g(x) < g(1) = 0$�Ee���_��.

        \end{proof}

\end{enumerate}



\section{NCQ�Qpe�_R} \label{sec:one-variable-differential}



\subsection{NCQ�Qpe�g<P�p�v$R�[} \label{subsec:extreme-point}

w $x_0$ �]�S$N�O/f&T:N@\�g'Y�\	�<P��g<P�p/f*jPWh.



\subsection{NCQ�Qpe�b�p�v$R�[} \label{subsec:inflection-point}



w $f''(x)$ /f&T(W $x_0$ �]�S$N�O_�S�勹p,g���S�N
N�S�[�N $f''(x) > 0$ �e:N�Q�Qpe�$f''(x) < 0$ �e:N�Q�Qpe��b�p/f�p.



\section{)R(u�l�RU\_Bl�gP�}



\[

  e^x = 1 + x + \dfrac{x^2}{2!} + \cdots + \dfrac{x^n}{n!} + \omicron(x^n)

\]



\[

  a^x = 1 + x \ln{a} + \dfrac{x^2}{2!} \ln^2{a} + \cdots + \dfrac{x^n}{n!} \ln^n{a} + \omicron(x^n)

\]



\[

  \sin{x} = x - \dfrac{x^3}{3!} + \dfrac{x^5}{5!} - \dfrac{x^7}{7!} + \cdots + (-1)^n \dfrac{x^{2n + 1}}{(2n + 1)!} + \omicron(x^{2n + 1})

\]



\[

  \cos{x} = 1 - \dfrac{x^2}{2!} + \dfrac{x^4}{4!} - \dfrac{x^6}{6!} + \cdots + (-1)^n \dfrac{x^{2n}}{(2n)!} + \omicron(x^{2n})

\]



\[

  \tan{x} = x + \dfrac{x^3}{3} + \dfrac{2x^5}{15} + \omicron(x^5)

\]



\[

  \ln{(1 + x)} = x - \dfrac{x^2}{2} + \dfrac{x^3}{3} - \dfrac{x^4}{4} + \cdots + (-1)^{n - 1} \dfrac{x^n}{n} + \omicron(x^n)

\]



\[

  \arcsin{x} = x + \dfrac{x^3}{6} + \dfrac{3x^5}{40} + \cdots

\]



\[

  \arctan{x} = x - \dfrac{x^3}{3} + \dfrac{x^5}{5} - \dfrac{x^7}{7} + \cdots + (-1)^{n - 1} \dfrac{x^{2n - 1}}{2n - 1} + \omicron(x^{2n - 1})

\]



\[

  (1 + x)^{1 / x} = e - \dfrac{ex}{2} + \dfrac{11ex^2}{24} - \dfrac{7ex^3}{16} + \dfrac{2447ex^4}{5760} + \omicron(x^4)

\]



\[

  (1 + x)^n = 1 + \dfrac{n}{1!} x + \dfrac{n (n - 1)}{2!} x^2 + \dfrac{n (n - 1) (n - 2)}{3!} x^3 + \cdots

\]



\[

  (1 + x)^{1 / n} = 1 + \dfrac{x}{n} - \dfrac{n - 1}{2!} \dfrac{x^2}{n^2} + \dfrac{(n - 1) (2n - 1)}{3!} \dfrac{x^3}{n^3} - \dfrac{(n - 1) (2n - 1) (3n - 1)}{4!} \dfrac{x^4}{n^4} + \cdots

\]



\section{�eyr�glQ_}



$$

  n! = \sqrt{2 \pi n} (\dfrac{n}{e})^n, n \to +\infty

$$



\subsection{NCQ�Qpeؚ6�Bl�[}



\subsubsection{ؚ6�Bl�[l�:NI{�kpeRBl�T}



\begin{example}

  $f(x) = \dfrac{1}{x^2 - x +1}$�Bl $f^{(2022)}(0)$.

\end{example}



\begin{solution}

  1u�z�e�TlQ_

  \[

    a^3 + b^3 = (a + b)(a^2 - ab + b^2)

  \]

  �_

  \[

    f(x) = \dfrac{1 + x}{1 + x^3} = \dfrac{1}{1 + x^3} + x\dfrac{1}{1 + x^3}

  \]

  �

  \[

    \dfrac{1}{1 + x^3} = \sum_{k = 0}^{\infty} (-x^3)^k, |x^3| < 1

  \]

  Ee

  \[

    f(x) = \sum_{k = 0}^{\infty} (-1)^k x^{3k} + \sum_{k = 0}^{\infty} (-1)^k x^{3k + 1}, |x^3| < 1

  \]

  Ee $f^{(2022)}(0)$ �^�:N $x^{2022}$ �v�|peXN�N $2022!$.



  �V:N $2022 = 3 \times 674 + 0$�@b�N $x^{2022}$ �v�|pe:N $(-1)^{674} = 1$�EeT{Hh:N $2022!$.

\end{solution}



d�)R(u�z�e�TlQ_Y�؏�S)R(u�z�e�]lQ_I{.



\subsubsection{ؚ6�Bl�[l�:N�l�RU\__}



\begin{example}

  $f(x) = x^2\ln{(1 + x)}$�Bl $f^{(n)}(0)$.

\end{example}



\begin{solution}

  �V:N

  \[

    \ln{(1 + x)} = \sum_{k = 1}^{\infty} \frac{(-1)^{k + 1} x^k}{k}

  \]

  Ee

  \[

    f(x) = x^2\ln{(1 + x)} = \sum_{k = 1}^{\infty} \frac{(-1)^{k + 1} x^{k + 2}}{k}

  \]

  Ee $f^{(n)}(0)$ �^:N $x^{n}$ �v�|peXN�N $n!$.



  1u $x^{k+2}$�S_ $k = n-2$ �e	g $x^n$��|pe:N $\frac{(-1)^{n-1}}{n-2}$�EeT{Hh:N

  \[

    f^{(n)}(0) = \frac{(-1)^{n-1}}{n-2} n!

  \]

\end{solution}



\section{NCQ�Qpe�gP�}



Bl $\lim_{x \to a} f(x)$ �e�����g $\lim_{x \to a^+} f(x)$ �T $\lim_{x \to a^-} f(x)$.



�

$$

  \lim_{x \to +\infty} f(x) = 0, \lim_{x \to +\infty} g(x) = +\infty

$$

R

$$

  \lim_{x \to +\infty} (1 + f(x))^{g(x)} = \exp{\lim_{x \to +\infty} f(x)g(x)}

$$



S_RP[bR�k:N9h_�v�R�Q�e��S\ՋRP[bR�k	gtS.



�	g $\lim_{x \to x_0} \dfrac{f(x)}{g(x)} = a$�R�S�Nl��Q:N(W $x = x_0$ �v�g*N���W�Q $f(x) = ag(x)$.



�[�N$\lim_{n \to +\infty} f(\dfrac{1}{n})$�
N�SO(u $x = \dfrac{1}{n}$ �v�Nbc��_{��OYu $\dfrac{1}{n}$ �vb__.



�[�N $\lim_{n \to +\infty} x_n = +\infty, \lim_{n \to +\infty} y_n = \lim_{n \to +\infty} z_n = a$�	g $\lim_{n \to +\infty} x_n [f(y_n) - f(z_n)] = \lim_{n \to +\infty} x_n (y_n - z_n) f'(\xi)$�vQ-N $\xi$ (W $y_n, z_n$ KN��.



\section{�gP��VRЏ�{X[(W'`} \label{sec:limit-arithmetic}



� $\lim_{x \to a} f(x) = A$, $\lim_{x \to a} g(x)$ 
NX[(W�R $\lim_{x \to a} f(x) + g(x)$ 
NX[(W.



S_ $A \neq 0$ �e��S	g$\lim_{x \to a} f(x)g(x)$ 
NX[(W�$A = 0$ �e
Nnx�[.



� $\lim_{x \to a} f(x)$ �T $\lim_{x \to a} g(x)$ GW
NX[(W�R $\lim_{x \to a} f(x) + g(x)$ �T $\lim_{x \to a} f(x)g(x)$ GW
Nnx�[.



\section{NCQ�Qpeޏ�~0�S�[0�S�_�v$R�[NsQ�|}



\subsection{NCQ�Qpeޏ�~�v$R�[}



� $\lim_{x \to x_0} f(x) = f(x_0)$�R�y $f(x)$ (W $x = x_0$ Yޏ�~.



� $\lim_{x \to x_0^-} f(x) = f(x_0)$�R�y $f(x)$ (W $x = x_0$ Y�]ޏ�~��Sޏ�~eu.



\subsection{NCQ�Qpe�S�[�v$R�[} \label{subsec:derivative-definition}



� $\lim_{x \to x_0} \dfrac{f(x) - f(x_0)}{x - x_0}$ X[(W�R�y $f(x)$ (W $x = x_0$ Y�S�[.



� $\lim_{x \to x_0^-} \dfrac{f(x) - f(x_0)}{x - x_0}$ X[(W�R�y $f(x)$ (W $x = x_0$ Y�]�S�[��S�S�[eu.



\subsection{NCQ�Qpe�S�_�v$R�[}



� $\Delta{y} = f(x_0 + \Delta{x}) - f(x_0) = A\Delta{x} + \omicron(\Delta{x})$�R�y $f(x)$ (W $x = x_0$ Y�S�_�N�_R $\mathrm{d}y = A\Delta{x} = A\mathrm{d}x$.



\subsection{NCQ�Qpeޏ�~0�S�[0�S�_�vsQ�|}



\[

  �S�_ \iff �S�[ \implies ޏ�~

\]



�lh�l�v�{4Yh�:y�e�l�c�Q.



\section{�[pe�gP��T�[pe�vsQ�|} \label{sec:derivative-limit}



$\lim_{x \to x_0} f'(x)$ X[(W��e�l$R�e $f(x)$ /f&T(W $x = x_0$ Yޏ�~.



� $\lim_{x \to x_0} f'(x) = A$�N $f(x)$ (W $x = x_0$ Yޏ�~�R $f'(x_0) = A$�&TR
NX[(W.



� $\lim_{x \to x_0} f'(x) = \infty$�R $f'(x_0)$ 
NX[(W.



� $\lim_{x \to x_0} f'(x)$ 
NX[(WN
N:N $\infty$�R���)R(u \cref{subsec:derivative-definition} -N�v$R�e.



\section{���e�p}



\subsection{,{N{|���e�p}



�S�S���e�p�$\lim_{x \to x_0} f(x)$ X[(W�FOvQN $f(x_0)$ 
N�vI{b $f(x_0)$ �e�[IN.



����e�p�$\lim_{x \to x_0^+} f(x)$ N $\lim_{x \to x_0^-} f(x)$ X[(WFO
N�vI{.



\subsection{,{�N{|���e�p}



$\lim_{x \to x_0^+} f(x)$ N $\lim_{x \to x_0^-} f(x)$ �NN
NX[(W.



\section{�f�s�T�f�sJS�_}



\[

  K = \dfrac{|x'y'' - x''y'|}{(x'^2 + y'^2)^{\frac{3}{2}}}

\]



\[

  R = \dfrac{1}{K}

\]



\section{�f�sJS�_�v�c�[}



� $x = x(t), y = y(t)$��N $(x - a)^2 + (y - b)^2 = R^2$ /fvQ(W�p $(x, y)$ �v�f�sW. �[�f�sWBlsQ�N $t$ �vN6�0�N6��[pe�

\[

  \left\{

  \begin{aligned}

    (x - a)x' + (y - b)y'                 & = 0 \\

    x'^2 + y'^2 + (x - a)x'' + (y - b)y'' & = 0 \\

  \end{aligned}

  \right.

\]

㉗_

\[

  \left\{

  \begin{aligned}

    x - a & = -\dfrac{x'^2 + y'^2}{x'y'' - x''y'} \cdot y' \\

    y - b & = \dfrac{x'^2 + y'^2}{x'y'' - x''y'} \cdot x'  \\

  \end{aligned}

  \right.

\]

Ee	g

\[

  R^2 = \dfrac{(x'^2 + y'^2)^3}{(x'y'' - x''y')^2}

\]



N,����vO��`OBlUS�p�v�f�s��S�N�v�c�NeQ�
Nb��v�ez�~.



\section{nя�~}



HQw���e�p��]�S�gP��NN:N�ewz $\implies$ Ŕ�vnя�~.



�Qw4ls^b�enя�~ $y = ax + b$�T7h����g $x \to +\infty$ �T $x \to -\infty$ $N*N�eT.



\section{NCQ�Qpe
N�[�yR}



\subsection{�S�QpeX[(W�[t}



\begin{itemize}

  \item � $f(x)$ (W $[a, b]$ 
Nޏ�~�R(W $[a, b]$ 
NX[(W�S�Qpe.

  \item � $f(x)$ (W $[a, b]$ 
N	g����e�p�R(W $[a, b]$ 
NN�[
NX[(W�S�Qpe.

\end{itemize}



$f(x)$ 
Nޏ�~�e��S�QpeX[(W'`N�[�yRX[(W'`�S�NT
N�vr^.



1u�S�Qpe�[IN, $F'(x) = f(x)$�Ee $F(x)$ ޏ�~.



\subsection{}Y(u�v_P[}



\[

  \int e^{ax} \cos{bx} \,\mathrm{d}x = e^{ax} \cdot \frac{a \cos{bx} + b \sin{bx}}{a^2 + b^2} + C

\]



\[

  \int e^{ax} \sin{bx} \,\mathrm{d}x = e^{ax} \cdot \frac{a \sin{bx} - b \cos{bx}}{a^2 + b^2} + C

\]



S_ $P(x)$ :NYy�_�e�NN	N*N_P[^�8^}Y(u.



\[

  \int P(x) e^{ax} \,\mathrm{d}x = e^{ax} \left( \frac{P}{a} - \frac{P'}{a^2} + \frac{P''}{a^3} - \cdots \right) + C

\]



\[

  \int P(x) \cos{ax} \,\mathrm{d}x = \cos{ax} \left( \frac{P'}{a^2} - \frac{P'''}{a^4} + \cdots \right) + \sin{ax} \left( \frac{P}{a} - \frac{P''}{a^3} + \cdots \right) + C

\]



\[

  \int P(x) \sin{ax} \,\mathrm{d}x = \sin{ax} \left( \frac{P'}{a^2} - \frac{P'''}{a^4} + \cdots \right) - \cos{ax} \left( \frac{P}{a} - \frac{P''}{a^3} + \cdots \right) + C

\]



\subsection{�YUO�_��b_R_Yy�_}



\begin{example}

  Bl $I = \int \dfrac{7x - 2}{(2x - 1)(x + 1)} \,\mathrm{d}x$.

\end{example}



\begin{solution}

  HQ\R_Yy�_�b_:N

  \[

    \dfrac{7x - 2}{(2x - 1)(x + 1)} = \dfrac{A}{2x - 1} + \dfrac{B}{x + 1}

  \]

  6qT(W_P[$N��T�eXN�N $2x - 1$ �_

  \[

    \dfrac{7x - 2}{x + 1} = A + \dfrac{2x - 1}{x + 1} B

  \]

  �Q�N $x = \dfrac{1}{2}$ �_ $A = 1$��Q(W_P[$N��T�eXN�N $x + 1$ �_

  \[

    \dfrac{7x - 2}{2x - 1} = \dfrac{x + 1}{2x - 1} A + B

  \]

  gT�N $x = -1$ �_ $B = 3$.



  Ee�S_:N

  \[

    \begin{aligned}

      I & = \int \left( \dfrac{1}{2x - 1} + \dfrac{3}{x + 1} \right) \,\mathrm{d}x \\

        & = \dfrac{1}{2} \ln{|2x - 1|} + 3\ln{|x + 1|} + C

    \end{aligned}

  \]

\end{solution}



\subsection{'k�blQ_(W�yR-N�v�^(u}



�N $y = e^{\mathrm{i}x}$�R	g

\[

  2\mathrm{i} \sin{x} = y - \frac{1}{y}, 2 \cos{x} = y + \frac{1}{y}

\]

\[

  2\mathrm{i} \sin{kx} = y^k - \frac{1}{y^k}, 2 \cos{kx} = y^k + \frac{1}{y^k}, k \in \mathbb{N}_0

\]



\begin{example}

  Bl $I = \int \cos^4{x} \,\mathrm{d}x$.

\end{example}



\begin{solution}

  �V:N

  \[

    \begin{aligned}

      \left(2\cos{x}\right)^4 & = \left(y + \frac{1}{y}\right)^4                              \\

                              & = y^4 + \frac{1}{y^4} + 4\left(y^2 + \frac{1}{y^2}\right) + 6 \\

                              & = 2 \cos{4x} + 8 \cos{2x} + 6

    \end{aligned}

  \]

  Ee

  \[

    \cos^4{x} = \frac{\cos{4x}}{8} + \frac{\cos{2x}}{2} + \frac{3}{8}

  \]

  \[

    I = \int \cos^4{x} \,\mathrm{d}x = \frac{\sin{4x}}{32} + \frac{\sin{2x}}{4} + \frac{3x}{8} + C

  \]

\end{solution}



�[�N
N�Q�sGYpe!kB^�vck&_�Qpe�yR^�8^}YYt. ��Q�s�NGYpe!kB^�N,�	c $\sin{x} \,\mathrm{d}x = -\mathrm{d}\cos{x}$ �v�e_Yt.



\subsection{9��f�yR�l} \label{subsec:feiman-integration}



� $f(x, t)$ (W $R[x \in [a, A], t \in [b, B]]$ �Q	g�[INNޏ�~�N $\dfrac{\partial{f}}{\partial{t}}$ (W $R$ �Qޏ�~�R	g

\[

  \dfrac{\mathrm{d}}{\mathrm{d}t} \int_{a}^{A} f(x, t) \,\mathrm{d}x = \int_{a}^{A} \dfrac{\partial{f}}{\partial{t}} \,\mathrm{d}x

\]



�fN,��`�QN�S_NP�:N $u(t)$�
NP�:N $v(t)$�N $t \in (b, B)$ �e	g

\[

  u(t) \in [a, A], v(t) \in [a, A]

\]

R	g

\[

  \dfrac{\mathrm{d}}{\mathrm{d}t} \int_{u(t)}^{v(t)} f(x, t) \,\mathrm{d}x = f(v(t), t) \cdot v'(t) - f(u(t), t) \cdot u'(t) + \int_{u(t)}^{v(t)} \dfrac{\partial{f}}{\partial{t}} \,\mathrm{d}x

\]



\begin{example}

  Bl $I = \int \dfrac{\mathrm{d}x}{(1 + x^2)^2}$.

\end{example}



\begin{solution}

  �g �

  \[

    f(x, t) = \int \dfrac{\mathrm{d}x}{t^2 + x^2}

  \]

  R

  \[

    \dfrac{\partial{f}}{\partial{t}} = \int \dfrac{-2t \,\mathrm{d}x}{(t^2 + x^2)^2}

  \]

  	g

  \[

    \left[\dfrac{\partial{f}}{\partial{t}}\right]_{t \to 1} = -2I

  \]

  Ee

  \[

    \begin{aligned}

      I & = -\dfrac{1}{2} \left[\dfrac{\partial{f}}{\partial{t}}\right]_{t \to 1}                                                                          \\

        & = -\dfrac{1}{2} \left[\dfrac{\partial}{\partial{t}} \int \dfrac{\mathrm{d}x}{t^2 + x^2} \right]_{t \to 1}                                        \\

        & = -\dfrac{1}{2} \left[\dfrac{\partial}{\partial{t}} \left( \dfrac{1}{t} \arctan{\dfrac{x}{t}} \right) \right]_{t \to 1}                          \\

        & = -\dfrac{1}{2} \left[ -\dfrac{1}{t^2} \arctan{\dfrac{x}{t}} - \dfrac{x}{t^3} \cdot \dfrac{1}{1 + \left(\dfrac{x}{t}\right)^2} \right]_{t \to 1} \\

        & = \dfrac{\arctan{x} + \dfrac{x}{1 + x^2}}{2} + C

    \end{aligned}

  \]

\end{solution}



�fY�Oo`�� \href{http://mathmarch.com/knowledge/0d9388502e0811eabe02815ed3256640}{dk���c}.



\section{NCQ�Qpe�[�yR}



\subsection{N,��Qpe�S8^�yR[ece'`�v$R�[}



�S8^�yR	g�NN$N�y�S���



\begin{itemize}

  \item \textbf{�ewzP��v�S8^�yR}    �yR
NNP��NN:N�ewz�$\int_{a}^{+\infty} f(x) \,\mathrm{d}x$0$\int_{-\infty}^{b} f(x) \,\mathrm{d}x$0$\int_{-\infty}^{+\infty} f(x) \,\mathrm{d}x$.

  \item \textbf{�eLu�Qpe�v�S8^�yR}    �yR:S���Q�g�p (Ut�p) �Qpe<P:N�ewz�$\int_{a}^{b} f(x) \,\mathrm{d}x$ N $\exists x_0 \in [a, b]$ O $f(x_0) = \infty$

\end{itemize}



S_6qُ$N�y�`�Q_N�S�NT�eb�z��cNeg�~�Q$R�[�e�l�



\paragraph{�ewzP�$R�[}�� $\lim_{x \to \infty} x^p f(x) = A$



\begin{itemize}

  \item �X[(W $p > 1$ O $A$ X[(W�R6e[e.

  \item �X[(W $p \leq 1$ O $A$ :N�ewzb^���pe�R�Sce.

\end{itemize}



\paragraph{Ut�p$R�[}�� $\lim_{x \to x_0^+} (x - x_0)^p f(x) = A$

\begin{itemize}

  \item �X[(W $p < 1$ O $A$ X[(W�R6e[e.

  \item �X[(W $p \geq 1$ O $A$ :N�ewzb^���pe�R�Sce.

\end{itemize}



S_6q`O؏���$R�[ $\lim_{x \to x_0^-} (x_0 - x)^p f(x) = A$��S	gS_TYGW6e[e�e�Mb��$R�[teSO6e[e.



\subsection{GYvP�Qpe�S8^�yR�v[ece'`}



� $\int_{-\infty}^{+\infty} f(x) \,\mathrm{d}x$ 6e[e�R

\[

  \int_{-\infty}^{+\infty} f(x) \,\mathrm{d}x =

  \begin{cases}

    0                                       & f(x) :NGY�Qpe \\

    2 \int_{0}^{+\infty} f(x) \,\mathrm{d}x & f(x) :NvP�Qpe

  \end{cases}

\]



�

\[

  \int_{-\infty}^{+\infty} f(x) \,\mathrm{d}x \coloneqq

  \lim\limits_{A \to +\infty, B \to -\infty} \int_{B}^{A} f(x) \,\mathrm{d}x

\]

X[(W�R

$$

  \lim\limits_{R \to +\infty} \int_{-R}^{R} f(x) \,\mathrm{d}x

$$

X[(W��SKN
NN�[b�z.



\subsection{�S8^�yR�T�v[ece'`}



\begin{table}[H]

  \small

  \centering

  \caption{�S8^�yR�T�v[ece'`} \label{fig:improper-integral-sum-convergence}

  \begin{tabular}{ccc}

    \hline

    $\int_{a}^{+\infty} f(x) \,\mathrm{d}x$ & $\int_{a}^{+\infty} g(x) \,\mathrm{d}x$ & $\int_{a}^{+\infty} [f(x) \pm g(x)] \,\mathrm{d}x$ \\

    \hline

    6e[e                                      & 6e[e                                      & 6e[e                                                 \\

    6e[e                                      & �Sce                                      & �Sce                                                 \\

    �Sce                                      & �Sce                                      & 
Nnx�[                                                \\

    \hline

    $\int_{-\infty}^{a} f(x) \,\mathrm{d}x$ & $\int_{a}^{+\infty} f(x) \,\mathrm{d}x$ & $\int_{-\infty}^{+\infty} f(x) \,\mathrm{d}x$      \\

    \hline

    6e[e                                      & 6e[e                                      & 6e[e                                                 \\

    --                                      & --                                      & �Sce                                                 \\

    \hline

  \end{tabular}

\end{table}



\subsection{	N҉�Qpe�v�yRyr'`}



�NN'`(�GW1u:S��͑�s�c�Q�



\[

  \int_{0}^{\pi} x f(\sin{x}) \,\mathrm{d}x = \dfrac{\pi}{2} \int_{0}^{\pi} f(\sin{x}) \,\mathrm{d}x

\]

\[

  \int_{0}^{\frac{\pi}{2}} f(\sin{x}) \,\mathrm{d}x = \int_{0}^{\frac{\pi}{2}} f(\cos{x}) \,\mathrm{d}x

\]



\subsection{Wallis lQ_}



\[

  \begin{aligned}

    I & = \int_{0}^{\frac{\pi}{2}} \sin^m{x} \cos^n{x} \,\mathrm{d}x \\

      & = \int_{0}^{\frac{\pi}{2}} \cos^m{x} \sin^n{x} \,\mathrm{d}x \\

      & =

    \begin{cases}

      \dfrac{(m - 1)!! (n - 1)!!}{(m + n)!!}                      & � m,n 
NGW:NvPpe \\

      \dfrac{(m - 1)!! (n - 1)!!}{(m + n)!!} \cdot \dfrac{\pi}{2} & � m,n GW:NvPpe

    \end{cases}

  \end{aligned}

\]



\subsection{hTg�Qpe�v�yRyr'`}



� $f(x)$ :NhTg:N $T$ �vޏ�~�Qpe�R	g

\[

  \int_{a}^{a + T} f(x)\,\mathrm{d}x = \int_{b}^{b + T} f(x)\,\mathrm{d}x

\]



\subsection{NCQ�Qpe�yRl�YCQ�Qpe�yR}



�[ $\rho = \rho(\theta)$�	gb��ylQ_

\[

  S = \frac{1}{2} \int_{\theta_1}^{\theta_2} \rho^2\,\mathrm{d}\theta

\]

EeS_Bl�YNb__�yR

\[

  I = \int_{a}^{b} f^2(\sin\theta, \cos\theta)\,\mathrm{d}\theta

\]

�S��

\[

  \rho = f(\sin\theta, \cos\theta)

\]

�N

\[x = \rho \cos\theta, y = \rho \sin\theta

\]

R	g

\[

  I = 2S

\]

l�S:N�N͑�yR�ffYt.



�[�N

\[

  I = \int_0^{+\infty} \frac{f(x)}{x}\,\mathrm{d}x

\]

�S(ulQ_

\[

  \frac{1}{x} = \int_0^{+\infty} e^{-xy}\,\mathrm{d}y

\]

�_

\[

  I = \int_0^{+\infty} \mathrm{d}y \cdot \int_0^{+\infty} f(x) e^{-xy}\,\mathrm{d}x

\]

yr+R�(u�N $f(x) = e^{ax}$ b $f(x) = a \sin{bx} + c \cos{dx}$ I{�`�Q. �S�� \href{https://www.bilibili.com/video/av1604217605}{dk���c} N \href{https://www.bilibili.com/video/av368640102}{dk���c}.



\subsection{�P]lpQ<\�yR (Frullani �yR) }



�� $f(x)$ (W $[0, +\infty)$ 
Nޏ�~�$a, b > 0$��[IN�YN�yR

\[

  I = \int_0^{+\infty} \frac{f(ax) - f(bx)}{x}\,\mathrm{d}x

\]



\begin{itemize}

  \item � $\lim_{x \to +\infty} f(x) = A$�R

        \[

          I = [f(0) - A] \ln{\frac{b}{a}}

        \]



  \item �X[(W $k > 0$ O�_ $\int_k^{+\infty} \frac{f(x)}{x}\,\mathrm{d}x$ 6e[e�R

        \[

          I = f(0) \ln{\frac{b}{a}}

        \]



  \item �X[(W $k > 0$ O�_ $\int_0^k \frac{f(x)}{x}\,\mathrm{d}x$ 6e[e�R

        \[

          I = -f(+\infty) \ln{\frac{b}{a}}

        \]

\end{itemize}



\subsection{�[�yR�v�^(u}



\subsubsection{s^b��Vb_b��y}



\begin{itemize}

  \item �[�N�v҉PWh�Qpe $y = f(x)$�b�S�Qpe $x = \varphi(y)$�b��ylQ_f�c�[.

  \item �gPWhN�	g

        \[

          S = \frac{1}{2} \int_a^b \rho^2(\theta)\,\mathrm{d}\theta

        \]

        �fY�Q�[�S�� \cref{subsec:jacobian-matrix}.

\end{itemize}



\subsubsection{s^b��f�~'_�}



\begin{itemize}

  \item �Spe�ezNlQ_f�c�[.

  \item �v҉PWh $y = f(x)$ �`b_�Sl�S:N�Speb__.

  \item �gPWhN	g

        \[

          l = \int_a^b \sqrt{\rho^2(\theta) + \rho'^2(\theta)}\,\mathrm{d}\theta

        \]

        �fY�Q�[�S�� \cref{sec:line-integral-and-surface-integral}.

\end{itemize}



\subsubsection{�el�SOSO�y}



f�c�[�v�`b_
NX���. ��N�gPWhb__h�:y��S�S�� \href{https://www.bilibili.com/video/av554024399}{dk���c} �S \cref{subsec:jacobian-matrix}.



�Q�҉�^ $\theta$ 	g�_\�SS $\mathrm{d}\theta$�vQ�el�SOSO�y�_CQ:N

\[

  \mathrm{d}V = \frac{2\pi}{3} \rho^3 \sin\theta\,\mathrm{d}\theta

\]



\begin{proof}

  b��y�_CQ:N\��eb_

  \[

    \mathrm{d}S = \rho\,\mathrm{d}\theta\,\mathrm{d}\rho

  \]

  �_SO�yCQ:N

  \[

    \mathrm{d}v = \mathrm{d}S \cdot 2\pi \rho \sin\theta

  \]

  ;`SO�y:N

  \[

    \mathrm{d}V = \int_0^\rho 2\pi \rho^2 \sin\theta\,\mathrm{d}\theta\,\mathrm{d}\rho = 2\pi \sin\theta \cdot \frac{\rho^3}{3} \,\mathrm{d}\theta

  \]

\end{proof}



\subsubsection{�el��fb�b��y}



�v҉PWh�`b_�[f�c�[��Spe�ez�`b_�S�v�cWY(u�Speb__��fY�S�� \cref{sec:line-integral-and-surface-integral}.



\section{YCQ�Qpe�_R}



\subsection{YCQ�QpeBl�[}



� $z = z(x, y)$�R

\[

  \mathrm{d}z = \dfrac{\partial z}{\partial x} \mathrm{d}x + \dfrac{\partial z}{\partial y} \mathrm{d}y

\]



� $z = z(u, v),\ u = u(x, y),\ v = v(x, y)$�R

\[

  \dfrac{\partial z}{\partial x} = \dfrac{\partial z}{\partial u} \dfrac{\partial u}{\partial x} + \dfrac{\partial z}{\partial v} \dfrac{\partial v}{\partial x}

\]

\[

  \dfrac{\partial z}{\partial y} = \dfrac{\partial z}{\partial u} \dfrac{\partial u}{\partial y} + \dfrac{\partial z}{\partial v} \dfrac{\partial v}{\partial y}

\]



� $I(x) = \int_{u(x)}^{v(x)} f(t) \mathrm{d}t$�R

\[

  \dfrac{\mathrm{d}I}{\mathrm{d}x} = f(v) v' - f(u) u'

\]



S_�yR�Q
NUS:NsQ�N $t$ �v�Qpe�e�����Nbc�Sϑ��la�_�T \cref{subsec:feiman-integration} -N_P[�v
NT.



\subsection{�NCQ�Qpe�gP�}



�

\[

  \lim_{(x, y) \to (x_0, y_0)} \dfrac{F(x, y)}{G(x, y)} = a

\]

R�S�Nl��Q:N(W $(x_0, y_0)$ �v�g*N���W�Q $F(x, y) = aG(x, y)$.



\begin{example}

  �]�w $f(x, y)$ (W�p $(0, 0)$ �v�S�_���W�Q	g�[IN�N�n��

  \[

    \lim_{x \to 0, y \to 0} \dfrac{f(x, y) - f(0, 0)}{x^2 + 1 - x\sin y} = -3

  \]

  R�Qpe $f(x, y)$ (W�p $(0, 0)$ Y/f \_\_\_\_�kX �g'Y<P  �g\<P b ^��g<P 	��p.

\end{example}



ُ���l��Q�N���HN $f(x, y) = -3(x^2 + 1 - x\sin y) + f(0, 0)$��Ndk�c�[O�_�Q^��g<P�p�v�~���FOb�NGP�[�vُ*N�Qpe/f	g��v��NeQ $(0, 0)$ O�S�s $f(0, 0)$ �e��sS
NX[(W��7h�vޏ�~�v0�N6��S�[�v�Qpe�n����a��[E�
N���v_N�c���N�S�_��b�Nُ*N�_/f	g��v�@b�N
N��l��Q.



\begin{solution}

  S_ $x \to 0, y \to 0$ �e

  \[

    x^2 + 1 - x\sin y \to 1 > 0

  \]

  Ee

  \[

    f(x, y) - f(0, 0) \to -3 < 0

  \]

  9hnc�g<P�p�[IN�(W $(0, 0)$ �v�g*N�S�_���W�Q	g $f(x, y) < f(0, 0)$�sS $f(0, 0)$ /f�g'Y<P�p.

\end{solution}



ُ�y��؏/f)R(u�[INZP}Y�N.



\subsection{�NCQ�Qpeޏ�~0OP�[X[(W0OP�[ޏ�~0�S�_�v$R�[�TsQ�|}



\subsection{�NCQ�Qpeޏ�~�v$R�[}



�

\[

  \lim_{(x, y) \to (x_0, y_0)} f(x, y) = f(x_0, y_0)

\]

R�y $f(x, y)$ (W $(x_0, y_0)$ Yޏ�~.



�S�N $x = \rho\cos\theta,\ y = \rho\sin\theta$ ۏL��Nbc�傁gP��~�gN $\theta$ �vsQ
NI{�N $f(x_0, y_0)$�R
Nޏ�~.



\subsection{�NCQ�QpeOP�[X[(W�v$R�[}



�

\[

  \lim_{x \to x_0} \dfrac{f(x, y_0) - f(x_0, y_0)}{x - x_0} X[(W \implies f'_x(x_0, y_0)X[(W

\]



�

\[

  \lim_{y \to y_0} \dfrac{f(x_0, y) - f(x_0, y_0)}{y - y_0} X[(W \implies f'_y(x_0, y_0)X[(W

\]



\subsection{�NCQ�QpeOP�[ޏ�~�v$R�[}



�

\[

  \lim_{(x, y) \to (x_0, y_0)} \dfrac{\partial f(x, y)}{\partial x} = f'_x(x_0, y_0)

\]

R�y $f(x, y)$ (W $(x_0, y_0)$ sQ�N $x$ �vOP�[ޏ�~.



Tt��[�N $y$ �vOP�[_N{|<O.



\subsection{�NCQ�Qpe�S�_�v$R�[}



� $f'_x(x_0, y_0)$ �T $f'_y(x_0, y_0)$ ��X[(W�R��g�gP�

\[

  \lim_{(\Delta x, \Delta y) \to (0, 0)} \dfrac{f(x_0 + \Delta x, y_0 + \Delta y) - f(x_0, y_0) - f'_x(x_0, y_0)\Delta x - f'_y(x_0, y_0)\Delta y}{\sqrt{\Delta x^2 + \Delta y^2}} = 0

\]

b�zR�y�S�_�&TR
N�S�_.



\subsection{�NCQ�Qpeޏ�~0OP�[X[(W0OP�[ޏ�~0�S�_�vsQ�|}



\[

  $N*NOP�[(W (x_0, y_0) ޏ�~ \implies

  \left\{

  \begin{aligned}

     & f(x, y) (W勹p�S�_                  \\

     & f(x, y) (W勹pޏ�~ \implies $N*NOP�[��X[(W

  \end{aligned}

  \right.

\]



*gh�l�{4Yh�:y�e�l�c�Q.



\subsection{YCQ�Qpe�g<P�p�v$R�[}



�� $z = F(x, y)$ (W $(x_0, y_0)$ YwQ	gޏ�~�N6�OP�[pe�N $f'_x = 0,\ f'_y = 0$, �� $f''_{xx} = A,\ f''_{xy} = B,\ f''_{yy} = C$�R�



\begin{itemize}

  \item � $AC - B^2 > 0$ N $A > 0$�:N�g\<P. $A < 0$�:N�g'Y<P.

  \item � $AC - B^2 < 0$�
N/f�g<P�p.

  \item � $AC - B^2 = 0$�
N��nx�[�(u�[IN$R�e.

\end{itemize}



\section{YCQ�Qpe�yR}



\subsection{n�bc�[�y'`}



� $D \subset \mathbb{R}^2$ N $\forall{(x, y) \in D}$ ��	g $(y, x) \in D$�R $D$ wQ	gn�bc�[�y'`.



\begin{example}

  Bl

  \[

    \begin{aligned}

      I & = \iint_{D} (\dfrac{x^2}{a^2} + \dfrac{y^2}{b^2}) \,\mathrm{d}x\mathrm{d}y \\

      D & = \{(x, y) | x^2 + y^2 \leqslant R^2\}

    \end{aligned}

  \]

\end{example}



\begin{solution}

  �V:N $D$ �n��n�bc�[�y'`�Ee

  \[

    I = \iint_{D} (\dfrac{y^2}{a^2} + \dfrac{x^2}{b^2}) \,\mathrm{d}x\mathrm{d}y

  \]

  Ee

  \[

    2I = (\dfrac{1}{a^2} + \dfrac{1}{b^2}) \iint_{D} (x^2 + y^2) \,\mathrm{d}x\mathrm{d}y

  \]

  Neu.

\end{solution}



S_6q�[�N�fؚ�~_N	g�v<O�~���dkYeu.



\subsection{Ŗ�S�k�w5�} \label{subsec:jacobian-matrix}



(WۏL��NCQ�Qpe�yR�eb�N�`ۏL�bcCQ�FO $\mathrm{d}x\mathrm{d}y$ �bcb�NHNbT�



b�Neg�c��NN�N $x = x(u, v)$, $y = y(u, v)$ 0R�^/f�NHNa`    /fُ7h�vN*N�Qpe $F$ \O(u�NTϑ $\begin{bmatrix}u\\v\end{bmatrix}$ T���Q $\begin{bmatrix}x(u, v)\\y(u, v)\end{bmatrix}$.



b�N�Q��g\:S�W
N��eQ�v�_\�S�RN���Q�v�_\�S�R�vQ�SƉ\O�~'`�Sbc��� $J = \begin{bmatrix}k_1 & k_3 \\ k_2 & k_4\end{bmatrix}$�b�Neg�c�[�<P.



\begin{solution}

  �[�N

  \[

    \begin{bmatrix}\mathrm{d}u \\ 0\end{bmatrix}

  \]

  	g

  \[

    J \begin{bmatrix}\mathrm{d}u \\ 0\end{bmatrix} = \begin{bmatrix}k_1\mathrm{d}u \\ k_2\mathrm{d}u\end{bmatrix}

  \]

  �ُ�S�[�^�Spe�ez�v�g\�eTTϑ�Ee

  \[

    k_1 \mathrm{d}u = \dfrac{\partial{x}}{\partial{u}} \cdot \mathrm{d}u \implies k_1 = \dfrac{\partial{x}}{\partial{u}}

  \]

  \[

    k_2 \mathrm{d}u = \dfrac{\partial{y}}{\partial{u}} \cdot \mathrm{d}u \implies k_2 = \dfrac{\partial{y}}{\partial{u}}

  \]

  Tt

  \[

    k_3 = \dfrac{\partial{x}}{\partial{v}}

  \]

  \[

    k_4 = \dfrac{\partial{y}}{\partial{v}}

  \]

  Ee

  \[

    J =

    \begin{bmatrix}

      \dfrac{\partial{x}}{\partial{u}} & \dfrac{\partial{x}}{\partial{v}} \\ \dfrac{\partial{y}}{\partial{u}} & \dfrac{\partial{y}}{\partial{v}}

    \end{bmatrix}

  \]

  �S	gb��y�_CQ�k�O

  \[

    \mathrm{abs}\det{J} \cdot \mathrm{d}u\mathrm{d}v = \,\mathrm{d}x\mathrm{d}y

  \]

\end{solution}



��`�S�� \href{https://www.bilibili.com/video/av82535620}{3Blue1Brown 
0Ŗ�S�k�w5�N�@b��vŖ�S�kL�R_0}. N1udk�S��s^b��v҉PWhl��gPWh�e $\mathrm{abs} \det{J}$ 1\I{�N $\rho$. ُ�[YCQ�Qpe_N/fb�z�v��Y	N�~�v҉PWhl��gPWh�e	g

\[

  \mathrm{abs} \det{J} = \rho^2\sin{\varphi}

\]



\section{�f�~�yR} \label{sec:line-integral-and-surface-integral}



勂��Q�[
N%N(���_Y�������SP��N�N�~0	N�~�`�Q. �R�Sgq \href{https://zhuanlan.zhihu.com/p/52347499?utm_psn=1791998179333525505}{dk���c}.



�fY�vsQƉ���

\begin{itemize}

  \item \href{https://www.bilibili.com/video/BV1dg4y1v7RP}{-N��S� �SƉS���<h�g�[t}

  \item \href{https://www.bilibili.com/video/BV19s41157Z4}{ce�^N�e�^���KQ�e旹ez�~0AmSOI{@b(u0R�v��}

  \item \href{https://www.bilibili.com/video/BV1a541127cX}{nabla �{P[ N�h�^0ce�^0�e�^}

\end{itemize}



\subsection{,{N{|�f�~�yR} \label{subsec:first-kind-line-integral}



,{N{|�f�~�yRN�yR�eT�esQ�ُ�(u�Nhϑ:W.



\begin{example}

  ���~�[�^ $\rho(x, y)$�R�~(�ϑ:N

  \[

    M = \int_{L} \rho(x, y) \,\mathrm{d}s

  \]

\end{example}



\begin{solution}

  �8^ZP�l/f~b0R�Spe $t$�O�_ $x = x(t), y = y(t)$�N�Oc�eTN�.



  R

  \[

    \mathrm{d}x = x'(t)\,\mathrm{d}t,\quad \mathrm{d}y = y'(t)\,\mathrm{d}t

  \]

  �N�

  \[

    \mathrm{d}s = \sqrt{\mathrm{d}x^2 + \mathrm{d}y^2} = \sqrt{x'^2 + y'^2}\,\mathrm{d}t

  \]

\end{solution}



�[�N�fؚ�~�v�`�Q�S�N{|�k�ct.



�[E�
N�,{N{|�f�~�yR_N�S�NbcCQ���� \href{https://open.163.com/newview/movie/free?pid=WHLGTKCPS&mid=XIAM50G16}{dk���c}�FOُ�u<O��
NOO��N�O�N�. �gPWh�Nbc؏/f�����v��f�~�_CQ $\mathrm{d}s = \sqrt{\rho^2 + \rho'^2}\,\mathrm{d}\theta$.



\subsection{,{�N{|�f�~�yR} \label{subsec:second-kind-line-integral}



,{�N{|�f�~�yRN�yR�eT	gsQ�ُ�(u�N�wϑ:W.



\begin{example}

  �]�w�R:W

  \[

    \vec{F}(x, y) = \begin{bmatrix}F_x(x, y) \\ F_y(x, y)\end{bmatrix}

  \]

  Bl�l�f�~ $\vec{L}$ �v�R $W_{\vec{L}}$.

\end{example}



\begin{solution}

  \[

    \begin{aligned}

      W_{\vec{L}} & = \int_{\vec{L}} \vec{F}\mathrm{d}\vec{s}        \\

                  & = \int_{\vec{L}} F_x\mathrm{d}x + F_y\mathrm{d}y

    \end{aligned}

  \]

  N>f6q	g $W_{\vec{L}} = -W_{-\vec{L}}$.

\end{solution}



{|<O�N,{N{|�f�~�yR��S�NO(u�Spe�ezBl�.



�[�N\��f�~��� \cref{subsec:nabla-curl-stokes}. �[�N^�\��f�~�e�hQ�f�~:N\��f�~�v^�[e��~O(u�Spe�ez�l.



\subsection{Nabla �{P[0�sϑ0�e�^0<h�glQ_N�eXbKQ�elQ_} \label{subsec:nabla-curl-stokes}



S_ $L$ :N�T�f�~�e���f�~�yRsS $\vec{F}$ �l@w�f�~ $L$ �v�sϑ. )R(u $W_{\vec{L}} = -W_{-\vec{L}}$�b�N�S�N\�s@bSb��WRrR:N�ewzY\b��W $\mathrm{d}\vec{S}$��N�	g

\[

  \oint_{\partial{D}}\vec{F}\mathrm{d}\vec{s} = \iint_{D} \vec{\nabla} \times \vec{F} \mathrm{d}\vec{S}

\]



ُ̑ $D$ N $\partial{D}$ �v�[INeu�ُ̑�v $\vec{\nabla}$ �y\O Nabla �{P[�_N�y�T�[��{P[.



\[

  \vec{\nabla} = \begin{bmatrix}\dfrac{\partial}{\partial{x_1}} \\ \vdots \\ \dfrac{\partial}{\partial{x_n}}\end{bmatrix}

\]



���c���vN�p/f $\vec{\nabla} \times \vec{F}$ 
N�[hQ/f�SXN��/fh�:yBl�e�^�va`�sS $\mathrm{rot} \vec{F}$�vQ�S�^�N�g�p҉�Rϑ�v'Y\.




NT�N�h�^�Tce�^��e�^
N���{US�v�c^0RvQ�N�~�^�FO/f�S	g(W	N�~-NvQ�~�g:NTϑ.



�[�N�NCQ�Qpeeg���_P[sS<h�glQ_

\begin{equation} \label{eq:green-theorem}

  \begin{aligned}

    \oint_{\partial{D}}\vec{F}\mathrm{d}\vec{s}

     & = \iint_{D} \vec{\nabla} \times \vec{F} \,\mathrm{d}x\mathrm{d}y \\

     & = \iint \limits_{D} \det{

      \begin{bmatrix}

        \dfrac{\partial}{\partial{x}} & \dfrac{\partial}{\partial{y}} \\ F_x & F_y

      \end{bmatrix}

    } \,\mathrm{d}x\mathrm{d}y

  \end{aligned}

\end{equation}



�[�N	NCQ�Qpeeg���_P[sS�eXbKQ�elQ_

\begin{equation} \label{eq:stokes-theorem}

  \oint_{\partial{D}}\vec{F}\mathrm{d}\vec{s} = \iint_{D} \vec{\nabla} \times \vec{F} \cdot \begin{bmatrix}\mathrm{d}y\mathrm{d}z \\ \mathrm{d}z\mathrm{d}x \\ \,\mathrm{d}x\mathrm{d}y\end{bmatrix}

\end{equation}



\subsection{�~�yR_�esQ�v$R�[}



�e�e:W�sS $\mathrm{rot} \vec{F} \equiv \vec{0}$ 1\	g_�esQ    �V:N҉�Rϑ�b�m�@b�N(W�~
N�v�yRR`:N 0.



\section{�fb��yR}



\subsection{,{N{|�fb��yR}



,{N{|�fb��yRN�yR�eT�esQ�ُ�(u�Nhϑ:W.



\begin{example}

  �O�Yb��[�^ $\rho(x, y, z)$�Rb�(�ϑ

  \[

    M = \iint_{\varSigma} \rho(x, y, z) \, \mathrm{d}S

  \]

  N,�eg���b� $\varSigma$ O�N $z = z(x, y)$ �vb__�~�Q��N�	g

  \[

    \mathrm{d}S = \sqrt{1 + \left( \dfrac{\partial{z}}{\partial{x}} \right)^2 + \left( \dfrac{\partial{z}}{\partial{y}} \right)^2} \, \mathrm{d}x \mathrm{d}y

  \]

\end{example}



\begin{solution}

  (W�fb� $z = z(x, y)$ 
N�S�p $(x_0, y_0, z_0)$�vQ(W $x$ �eT
N	g�g\�S�R $\mathrm{d}x$�R�S�RTϑ:N

  \[

    \vec{v_1} = \left(1, 0, \dfrac{\partial{z}}{\partial{x}} \right) \mathrm{d}x

  \]

  Tt�$y$ �eT	g

  \[

    \vec{v_2} = \left(0, 1, \dfrac{\partial{z}}{\partial{y}} \right) \mathrm{d}y

  \]

  �S�_b��y�_CQsS$N*NTϑ�v�S�y�v!j��sS

  \[

    \mathrm{d}S = |\vec{v_1} \times \vec{v_2}|

  \]

  �[�NvQ�N�`�Q_NN7h.

\end{solution}



ُ̑�QY�cN*N�b�]�傫��y�Qpe $\rho(x, y, z) = F(x, y) \cdot z$ N $z \dfrac{\partial{z}}{\partial{x}}$0$z \dfrac{\partial{z}}{\partial{y}}$ �[f���{�R	g

\[

  M = \iint_{\varSigma} F(x,y) z\, \mathrm{d}S = \iint_{D} F(x, y) \sqrt{z^2 + \left(z \dfrac{\partial{z}}{\partial{x}}\right)^2 + \left(z \dfrac{\partial{z}}{\partial{y}}\right)^2} \, \mathrm{d}x \mathrm{d}y

\]



ُ(Wtb�0%�b�
N�gvQ}Y(u.



\subsection{,{N{|�fb��yR�v�Spe�ez�l}



���y�fb�:N $F(x, y, z) = 0$��N

\[

  x = x(u, v), y = y(u, v), z = z(u, v)

\]

�n���_N�OT.



vQ(W $u$ �eT
N	g�g\�S�R $\mathrm{d}u$�R�S�RTϑ:N

\[

  \vec{v_1} = \left(\dfrac{\partial{x}}{\partial{u}}, \dfrac{\partial{y}}{\partial{u}}, \dfrac{\partial{z}}{\partial{u}} \right) \mathrm{d}u

\]

Tt�$v$ �eT	g

\[

  \vec{v_2} = \left(\dfrac{\partial{x}}{\partial{v}}, \dfrac{\partial{y}}{\partial{v}}, \dfrac{\partial{z}}{\partial{v}} \right) \mathrm{d}v

\]

b��y�_CQ:N

\[

  \mathrm{d}S = |\vec{v_1} \times \vec{v_2}| = \left| \left( \dfrac{\partial{(y, z)}}{\partial{(u, v)}}, \dfrac{\partial{(z, x)}}{\partial{(u, v)}}, \dfrac{\partial{(x, y)}}{\partial{(u, v)}} \right) \mathrm{d}u \mathrm{d}v \right|

\]

vQ-N

\[

  \dfrac{\partial{(y, z)}}{\partial{(u, v)}} =

  \begin{vmatrix}

    \dfrac{\partial{y}}{\partial{u}} & \dfrac{\partial{y}}{\partial{v}} \\

    \dfrac{\partial{z}}{\partial{u}} & \dfrac{\partial{z}}{\partial{v}}

  \end{vmatrix}

\]



\begin{example}

  Bl

  \[

    I = \iint_{\varSigma} \sqrt{\dfrac{x^2}{a^4} + \dfrac{y^2}{b^4} + \dfrac{z^2}{c^4}} \, \mathrm{d}S

  \]

  vQ-N

  \[

    \varSigma: \dfrac{x^2}{a^2} + \dfrac{y^2}{b^2} + \dfrac{z^2}{c^2} = 1

  \]

\end{example}



\begin{solution}

  �N

  \[

    x = a\sin{\varphi}\cos{\theta}, y = b\sin{\varphi}\sin{\theta}, z = c\cos{\varphi}

  \]

  	g

  \[

    \mathrm{d}S = |\vec{v_1} \times \vec{v_2}| = abc \sin{\varphi} \sqrt{\dfrac{x^2}{a^4} + \dfrac{y^2}{b^4} + \dfrac{z^2}{c^4}} \, \mathrm{d}\varphi \mathrm{d}\theta

  \]

  sS

  \[

    \begin{aligned}

      I & = \int_{0}^{2\pi} \mathrm{d}\theta \int_{0}^{\pi} abc \left(\dfrac{\sin^3{\varphi} \cos^2{\theta}}{a^2} + \dfrac{\sin^3{\varphi} \sin^2{\theta}}{b^2} + \dfrac{\sin{\varphi} \cos^2{\varphi}}{c^2} \right) \mathrm{d}\varphi                                            \\

        & = abc \left( \dfrac{2 \cdot \dfrac{2!!}{3!!} \cdot 4 \cdot \dfrac{1!!}{2!!} \cdot \dfrac{\pi}{2}}{a^2} + \dfrac{2 \cdot \dfrac{2!!}{3!!} \cdot 4 \cdot \dfrac{1!!}{2!!} \cdot \dfrac{\pi}{2}}{b^2} + \dfrac{2 \cdot \dfrac{0!! \cdot 1!!}{3!!} \cdot 2\pi}{c^2} \right) \\

        & = \dfrac{4\pi abc}{3} \left( \dfrac{1}{a^2} + \dfrac{1}{b^2} + \dfrac{1}{c^2} \right)

    \end{aligned}

  \]

\end{solution}



\subsection{,{�N{|�fb��yR}



,{�N{|�fb��yRN�yR�eT	gsQ�ُ�(u�N�wϑ:W.



\begin{example}

  �]�w�x:W $\vec{B}(x, y, z) = \begin{bmatrix}B_x(x, y, z) \\ B_y(x, y, z) \\ B_z(x, y, z)\end{bmatrix}$�R�x�ϑ:N

  \[

    \varPhi_{\vec{S}} = \iint_{\vec{S}} \vec{B} \cdot \mathrm{d}\vec{S} = \iint_{\vec{S}} B_x \, \mathrm{d}y\mathrm{d}z + B_y \, \mathrm{d}z\mathrm{d}x + B_z \, \mathrm{d}x\mathrm{d}y

  \]

  N>f6q	g $\varPhi_{\vec{S}} = -\varPhi_{-\vec{S}}$.

\end{example}



\subsection{,{�N{|�fb��yR�v�Spe�ez�l}



Tt�	g�

\[

  \mathrm{d}\vec{S} = \vec{v_1} \times \vec{v_2} = \left( \dfrac{\partial{(y, z)}}{\partial{(u, v)}}, \dfrac{\partial{(z, x)}}{\partial{(u, v)}}, \dfrac{\partial{(x, y)}}{\partial{(u, v)}} \right) \mathrm{d}u\mathrm{d}v

\]

�Vdk�

\[

  \vec{B} \cdot \mathrm{d}\vec{S} = \left( B_x \dfrac{\partial{(y, z)}}{\partial{(u, v)}} + B_y \dfrac{\partial{(z, x)}}{\partial{(u, v)}} + B_z \dfrac{\partial{(x, y)}}{\partial{(u, v)}} \right) \mathrm{d}u\mathrm{d}v

\]



b�Negwyr�O�



�	gT�fb� $\vec{S}$ �N $z = z(x, y)$ �vb__�~�Q�vQ(W $xOy$ b�
N�v�bq_:N $D$�R	g�

\[

  \vec{B} \cdot \mathrm{d}\vec{S} = \left( -\dfrac{\partial{z}}{\partial{x}} B_x -\dfrac{\partial{z}}{\partial{y}} B_y + B_z \right) (\pm \mathrm{d}x\mathrm{d}y)

\]

vQ-N�S_ $\vec{S}$ N $z$ t�TT�e�S $+$, �SKN�S $-$.



\subsection{�ϑ0ce�^Nؚ�elQ_}




N�P�sϑ	g��f�~�v��Bl, ,{�N{|�fb��yRsS�y:N�ϑ�b�Nxvz��fb�
N�v�ϑ.



ce�^h�:y:N�

\[

  \vec{\nabla} \cdot \vec{F} = \mathrm{div} \vec{F} = \dfrac{\partial{F_x}}{\partial{x}} + \dfrac{\partial{F_y}}{\partial{y}} + \dfrac{\partial{F_z}}{\partial{z}}

\]

vQ�S f�g�pAm�Qir(�NAmeQir(��vY\.



)R(u $\varPhi_{\vec{S}} = -\varPhi_{-\vec{S}}$�b�N�S�N\�fb�@bS�VSO�WRrR:N�ewzY*N\SOCQ $\mathrm{d}V$��N��_0R�

\[

  \begin{aligned}

    I & = \oiint_{\partial{V}} \vec{F} \cdot \mathrm{d}\vec{S}                                                                                               \\

      & = \iiint_{V} \vec{\nabla} \cdot \vec{F} \, \mathrm{d}V                                                                                               \\

      & = \iiint_{V} \left( \dfrac{\partial{F_x}}{\partial{x}} + \dfrac{\partial{F_y}}{\partial{y}} + \dfrac{\partial{F_z}}{\partial{z}} \right) \mathrm{d}V

  \end{aligned}

\]



ُ1\/fؚ�elQ_.



\subsection{b��yR_�esQ�v$R�[}



��wϑ:W�ece�sS $\mathrm{div} \vec{F} \equiv 0$�Rb��yR_�esQ. �V:N�Na\��fb�
N�vAmeQ�TAm�QO�b�m�@b�N(Wb�
N�v�yRR`:N 0.




��\chapter{TϑN�QUO}



\section{Tϑ�v�QUOaIN}



Tϑ�pXN $\vec{a} \cdot \vec{b} = |\vec{a}| |\vec{b}| \cos{\theta}$�vQ-N $\theta$ :NTϑ $\vec{a}, \vec{b}$ @b9Y҉�v'Y\��QUOaINeu.



�N�~Tϑ*O�SXN $|\vec{a} \times \vec{b}| = |\vec{a}| |\vec{b}| \sin{\theta}$�vQ-N $\theta$ :NTϑ $\vec{a}, \vec{b}$ @b9Y҉�v'Y\��QUOaIN:NTϑ $\vec{a}, \vec{b}$ @b9Ys^L��V��b_b��y.



$NTϑ $\vec{a}, \vec{b}$ 9Y҉:N�҉�vEQ��ag�N:N $\vec{a} \cdot \vec{b} > 0, |\vec{a} \times \vec{b}| \ne 0$.



$NTϑ $\vec{a}, \vec{b}$ 9Y҉:N��҉�vEQ��ag�N:N $\vec{a} \cdot \vec{b} < 0, |\vec{a} \times \vec{b}| \ne 0$.



\section{�SXN}



\subsection{�N�~}



�SXN�^�;N��/f	N�~Tϑ���v���{�b�NHQ���vQ(W�N�~-N�v�^(u.



�N�~Tϑ*O�SXN $|\vec{a} \times \vec{b}| = |\vec{a}| |\vec{b}| \sin{\theta}$.



�

\[

  \begin{aligned}

    \vec{a} & = \begin{bmatrix}m \\ n \end{bmatrix} = (m, n) \\

    \vec{b} & = \begin{bmatrix}r \\ s \end{bmatrix} = (r, s)

  \end{aligned}

\]

R

\[

  |\vec{a} \times \vec{b}| = |\det(\begin{bmatrix}m & r \\ n & s \end{bmatrix})| = |\begin{vmatrix}m & r \\ n & s \end{vmatrix}| = |ms - nr|

\]



vQ-N $\det()$ (uegBl�w5�L�R_�wQSO�Sw�~'`�Npe�vsQ�wƋ.



�Ǐ
N���N�~�b�N�S�N�_�_Bl�Q$NTϑ@b9Ys^L��V��b_�vb��y.



\subsection{	N�~}



	N�~Tϑ�SXN $\vec{a} \times \vec{b}$ @b���Q�v/fN*N�e�vTϑ $\vec{n}$.



vQ-N $|\vec{n}|$ I{�NTϑ $\vec{a}, \vec{b}$ @b9Ys^L��V��b_b��y�$\vec{n}$ :N�s^L��V��b_�v�lTϑ.



N�eT�S�SKb�c�_gb��6ew��e
Tc�T\c�-NccT��]�dk�e $\vec{a}$ :Nߘc�$\vec{b}$ :N-Nc�$\vec{n}$ :N'Y�bc.



\begin{figure}[H]

  \small

  \centering

  \includegraphics[width=0.3\textwidth]{vec/3d-cross-product.png}

  \caption{	N�~Tϑ�SXN:ya�V} \label{fig:3d-cross-product}

\end{figure}



dkY�S�la0R $\vec{a} \times \vec{b} = -\vec{b} \times \vec{a}$



� $\vec{a} = \begin{bmatrix} a_1 \\ a_2 \\ a_3 \end{bmatrix} = (a_1, a_2, a_3), \vec{b} = \begin{bmatrix} b_1 \\ b_2 \\ b_3 \end{bmatrix} = (b_1, b_2, b_3)$



Rb�NRh� $\begin{vmatrix} a_1 & a_2 & a_3 \\ b_1 & b_2 & b_3 \end{vmatrix}$



R $\vec{a} \times \vec{b} = (n_1, n_2, n_3)$ -N



$n_1$ :N!cOO@bRh�,{NRT�vL�R_ $\begin{vmatrix} a_2 & a_3 \\ b_2 & b_3 \end{vmatrix}$



$n_2$ :N!cOO@bRh�,{�NRT�vL�R_ $\textcolor{red}{-}\begin{vmatrix} a_1 & a_3 \\ b_1 & b_3 \end{vmatrix}$



$n_3$ :N!cOO@bRh�,{	NRT�vL�R_ $\begin{vmatrix} a_1 & a_2 \\ b_1 & b_2 \end{vmatrix}$



�Ǐ
N���N�~�b�N�S�N�_�_Bl�Q�lTϑ



\section{I{�T0�]0�y0FU0s^�e�T�~}



\subsection{I{�T�~}



� $a + b \equiv 1$



\subsection{I{�]�~}



� $a - b \equiv 1$



\subsection{I{�y�~}



� $ab \equiv 1$



\subsection{I{FU�~}



� $\dfrac{a}{b} \equiv 1$



\subsection{I{s^�e�T�~}



� $a^2 + b^2 \equiv 1$



\section{�zSO�QUO-N�v�^(u}



\subsection{BlSO�y}



�[�NqQw��p�v	N�~Tϑ $\vec{a}, \vec{b}, \vec{c}$ 	g�N�Tϑ@b�Vb�vs^L�mQb�SO�vSO�y

\[

  V_1 = |(\vec{a} \times \vec{b}) \cdot \vec{c}|

\]

@b�Vb�v	N�h%��vSO�y

\[

  V_2 = \dfrac{V_1}{6}

\]



\subsection{Bl�Nb�҉}



(WBl�Y $E-AB-F$ �v�Nb�҉YO&_<P�e�1u�N�lTϑBl�l
NT��[��lTϑ9Y҉^��Nb�҉����;N$R�evQck���[f_�S����s�N�~N�y�e�l�MQ���



\begin{example}

  Bl \cref{fig:dihedral-angle} -N $E-AB-F$ �Nb�҉YO&_.

\end{example}



\begin{solution}

  Rs^b� $ABE$ N*N�lTϑ $\vec{n}_1 = \overrightarrow{AB} \times \overrightarrow{AE}$�s^b� $ABF$ N*N�lTϑ $\vec{n}_2 = \overrightarrow{AB} \times \overrightarrow{AF}$.



  \begin{figure}[H]

    \small

    \centering

    \begin{tikzpicture}

      \draw (0, 0) node[above left] {$A$};

      \draw (1, -2) node[right] {$B$};

      \draw (1.75, 0.25) node[right] {$E$};

      \draw (-2, -1) node[left] {$F$};

      \draw[->] (0, 0) -- (1, -2);

      \draw[->] (0, 0) -- (1.75, 0.25);

      \draw[->] (0, 0) -- (-2, -1);



      \draw (0, 0) -- (-2.5, 0) -- (-1.5, -2) -- (1, -2);

      \draw (0, 0) -- (1, 1.75) -- (2, -0.25) -- (1, -2);



      \draw (-0.75, -0.75) -- (-0.75, 0.25);

      \draw[->] (-0.75, -2) -- (-0.75, -2.5) node[right] {$\vec{n}_1$};



      \draw[->] (1, 0) -- (0, 4/7) node[above] {$\vec{n}_2$};

      \draw (121/65, -32/65) -- (2.25, -5/7);

    \end{tikzpicture}

    \caption{�Nb�҉} \label{fig:dihedral-angle}

  \end{figure}



  R1u \cref{fig:dihedral-angle-side} f�_ $E-AB-F$ sS $\vec{n}_1, \vec{n}_2$ �v9Y҉.



  \begin{figure}[H]

    \small

    \centering

    \begin{tikzpicture}

      \draw (0, 0) node[below right] {$B(A)$};

      \draw (-2, 0) node[below left] {$F$};

      \draw (1, 1.75) node[right] {$E$};



      \draw (0, 0) -- (-2.5, 0);

      \draw (0, 0) -- (1, 1.75);



      \draw[->] (-9/8, 0.5) -- (-9/8, -0.5) node[right] {$\vec{n}_1$};

      \draw[->] (1, 33/56) -- (0, 65/56) node[above right] {$\vec{n}_2$};

    \end{tikzpicture}

    \caption{�Nb�҉�OƉ�V} \label{fig:dihedral-angle-side}

  \end{figure}



  Ee�SBl�_�Nb�҉YO&_<P.

\end{solution}



\section{�v�~�ez�v�QUOaIN} \label{sec:line-equation-geometry}



N,�_h�:y$Ns^b��v�N�~�

\[

  \left\{

  \begin{aligned}

    A_1x + B_1y + C_1z + D_1 & = 0 \\

    A_2x + B_2y + C_2z + D_2 & = 0

  \end{aligned}

  \right.

\]



vQ�eTTϑN$N*Ns^b��v�lTTϑ�W�v�Ee�S�N $\vec{l} = \vec{n}_1 \times \vec{n}_2$.



�[�y_�T�Spe_���_�[ft��eu.



\section{�p0Rs^b�ݍ�ylQ_�v�QUOaIN}



$A = (x_0, y_0, z_0)$ 0Rs^b� $Ax + By + Cz + D = 0$ �vݍ�y:N

\[

  d = \frac{|Ax_0 + By_0 + Cz_0 + D|}{\sqrt{A^2 + B^2 + C^2}}

\]



�Ss^b�
NN�p $B = (x_1, y_1, z_1)$�Ee>f6q $\overrightarrow{BA}$ (W�lTϑ�v�bq_sS:Nݍ�y�Ee

\[

  d = \frac{|(A, B, C) \cdot (x_1 - x_0, y_1 - y_0, z_1 - z_0)|}{|(A, B, C)|}

\]

Neu.



\section{�p0R�v�~ݍ�ylQ_�v�QUOaIN}



$A = (x_0, y_0, z_0)$ 0R�Ǐ $B = (x_1, y_1, z_1)$��eTTϑ:N $\vec{n}$ �v�v�~�vݍ�y�

\[

  d = \frac{|\overrightarrow{BA} \times \vec{n}|}{|\vec{n}|}

\]



�V:N $\overrightarrow{BA}$ �T $\vec{n}$ @b9Ys^L��V��b_b��y:N $|\overrightarrow{BA} \times \vec{n}|$�Eed��N�^ $|\vec{n}|$ �_ؚ $d$.



\section{zz���f�~�vR�~N�ls^b��ez�v�QUOaIN}



\subsection{�N�Spe�ez�~�Q�ezz���f�~�v�QUOaIN}



��f�~�v�Spe�ez:N�

\[

  \left\{

  \begin{aligned}

    x & = x(t) \\

    y & = y(t) \\

    z & = z(t)

  \end{aligned}

  \right.

\]



RvQ
NN�p�vRTϑ:N $\vec{l} = (x', y', z')$.



�~TirtaINegw�\ $t$ Ɖ:N�e�����HNT*N�eT
NMOnsQ�N�e���v�[pe�Y $x'(t)$ 1\/f勹eT
N�v��^�	g $\vec{l_x} = (x', 0, 0)$. �S�V:N��^/f�wϑ�Ee�S�R�_0R $\vec{l}$.



	g�NRTϑ���HNR�~�T�ls^b��v�ez�v�QUOaIN�_f>f�N�eu.



\subsection{�N�ez�~�~�Q�ezz���f�~�v�QUOaIN}



��f�~�v�ez�~:N�

\[

  \left\{

  \begin{aligned}

    F(x, y, z) & = 0 \\

    G(x, y, z) & = 0

  \end{aligned}

  \right.

\]



vQh�:y$N�fb��v�N�~. b�N�Q��N�~
NN�p $(x_0, y_0, z_0)$�(W $F$ 
N	g�lTϑ $\vec{n}_1$�wQSOBl�l�� \cref{subsec:implicit-surface-geometry}.



(W $G$ 
N	g�lTϑ $\vec{n}_2$�>f6q�N�~
N勹p�v�eTTϑ $\vec{l} = \vec{n}_1 \times \vec{n}_2$.



�S�N�la0R�\cref{sec:line-equation-geometry} 1\/f,g���vN*Nyr�O.



\section{zz���fb��vRs^b�N�l�~�ez�v�QUOaIN}



\subsection{�N���Qpe�~�Q�e} \label{subsec:implicit-surface-geometry}



��fb��ez:N $F(x, y, z) = 0$�



�Q��fb�
NN�p $(x_0, y_0, z_0)$ �~Ǐ�_\�S�R0R�fb�
N�SN�p

\[

  (x_0 + \Delta x, y_0 + \Delta y, z_0 + \Delta z)

\]

1uhQ�XϑlQ_	g�

\[

  F(x_0 + \Delta x, y_0 + \Delta y, z_0 + \Delta z) = F(x_0, y_0, z_0) + \frac{\partial F}{\partial x} \Delta x + \frac{\partial F}{\partial y} \Delta y + \frac{\partial F}{\partial z} \Delta z = 0

\]

� $F(x_0, y_0, z_0) = 0$�Ee

\[

  \frac{\partial F}{\partial x} \Delta x + \frac{\partial F}{\partial y} \Delta y + \frac{\partial F}{\partial z} \Delta z = 0

\]

�SƉ:N$N*NTϑ

\[

  \vec{n} = \left( \frac{\partial F}{\partial x}, \frac{\partial F}{\partial y}, \frac{\partial F}{\partial z} \right)

\]

N

\[

  \vec{l} = (\Delta x, \Delta y, \Delta z)

\]

�v�p�y��f$NTϑ�W�v.



�S1u�N $\vec{l}$ /fb�N�Na�S�v�� $\vec{n}$ �[@b	g $\vec{l}$ ���W�v��V� $\vec{n}$ 1\/f勹p�v�lTϑ.



dkY $\vec{n}$ _N���y:N�h�^�)R(u \cref{subsec:nabla-curl-stokes} -N�v�wƋ�b�N\vQ��\O $\vec{\nabla}F$.



�lȃُ�v $F$ N $\vec{\nabla}$ �vЏ�{NTϑpeXN{|<O�vQ�~�g/fN*NTϑ.



\subsection{�N>f�Qpe�~�Q�e} \label{subsec:explicit-surface-geometry}



��fb��N $z = f(x, y)$ �vb__�~�Q�



)R(u \cref{subsec:implicit-surface-geometry} �v�~����S $F(x, y, z) = f(x, y) - z = 0$ sS�S.



\section{�~'`�Npe}



(Wck_ۏeQ�~'`�NpeُNUSCQKNMR���‰w�NNƉ��D��e�



\begin{itemize}

  \item \href{https://b23.tv/I25Kyqf}{�~'`�Npe�v,g(�}

  \item \href{https://www.bilibili.com/video/BV1wu411T7dj}{�e�u�~�N}

  \item \href{https://www.bilibili.com/video/BV12N4y1H7Rn}{�N!k�Wvz�z/f*NeU��N!k�W�v�QUOaIN}

\end{itemize}



b�N�N*N�w5� $A$ �[�^N*N�~'`�Sbc�$A\vec{x} = \vec{y}$ sS $\vec{x}$ �~ $A$ �v�~'`�Sbc�_0R $\vec{y}$�� $AB = C$ Rh�:y\ $B$ �v�kN*NRTϑ�~ $A$ �v�~'`�Sbc�_0R $C$.



>f6q��[ $A$ @b	g�S���v���QTϑ $\vec{y}$ �gb�NN*NƖT�b�N(Wdk�{US0W\ُ*NƖT�y:N $A$ �v \href{https://zh.wikipedia.org/wiki/%E5%90%91%E9%87%8F%E7%A9%BA%E9%97%B4}{Tϑzz��}���b@b	g�S���v��eQTϑ�vƖT�y:N�STϑzz��.



\section{�y}



vQ�QUO+TIN��� \href{https://b23.tv/I25Kyqf}{�~'`�Npe�v,g(�}�b�N(Wdk�W@x
N�ʑN�N8^(u�~���



\begin{itemize}

  \item $A$ �v�y $= A$ �vL��y $= A$ �vR�y



        \begin{proof}

          �S�w5� $A_{m \times n}$��� $A$ R�y:N $r$�Ee $A$ �vRzz���v�~�^/f $r$.



          �N $\vec{c_1}, \cdots, \vec{c_r}$ /f $A$ �vRzz���vN�~�W��gb�w5� $C_{m \times r}$�O�_ $A$ �v�k*NRTϑ/f $C$ �v $r$ *NRTϑ�v�~'`�~T�sSX[(W�w5� $P$ O�_ $A = CP$.



          ��HN	g $A$ �v�k*NL�Tϑ��/f $P$ �vL�Tϑ�v�~'`�~T�Ee $A$ L�Tϑ�~�vTϑzz��(W $P$ �vL�Tϑ�~�vTϑzz���Q�Ee $A$ �vL��y $\leqslant$ $P$ �vL��y $\leqslant r = A$ �vR�y.



          �Q�Q� $A^T$�R	g $A$ �vR�y $= A^T$ �vL��y $\leqslant A^T$ �vR�y $= A$ �vL��y.



          �~
NEe$N��vI{.

        \end{proof}



  \item �w5� $A$ N $B$ I{�N $\iff$ �w5� $A$ N $B$ T�WN $r(A) = r(B)$



        \begin{proof}

          �S,g�v�[IN $B_{m \times n} = Q_{m \times m}A_{m \times n}P_{n \times n}$ N $P, Q$ GW�S�vQ�[1\�f+T�Nُ�p�$N��[�^�NTNTϑzz��.

        \end{proof}



  \item $0 \leqslant r(A_{m \times n}) \leqslant \min\{m, n\}$



        \begin{proof}

          �gvQ>f6q��yT�e��Tϑ@b(W�v�~�^�TTϑ�v*NpeP�6R.

        \end{proof}



  \item $r(A) = r(A^T) = r(AA^T) = r(A^TA)$



        \begin{proof}

          ,{N*NI{�S>f6q���f $A\vec{x} = \vec{0}$0$A^TA\vec{x} = \vec{0}$ T�sS��f,{�N*NI{_�Tt��f,{	N*N.

        \end{proof}

\end{itemize}



\section{�W@x��|}



b�N� $A \vec{x} = \vec{0}$ �[E�
N1\/fN*N $\vec{x}$ Tϑ�~Ǐ�w5� $A$ �v�Sbc�Sb�NN*N��Tϑ�b�NBl㉄vǏzck/fBl`7h�v $\vec{x}$ O�~�Sbc�Sb��Tϑ.



�Y�g $A$ /f�n�y�v��f $A$ �v�Sbc�l	g�Q�sM��~�ُ�f@b	g�S^���Tϑ�SbcT��/f^���Tϑ���HNb�N�v��S	g����
N/f��*N�	�.



�SKNN�[X[(WN*Nzz��̑�v@b	gTϑ���Sb�N��Tϑ�b�NBl�W@x��|1\/f(WBlُ*Nzz���v�W�^	�.



\begin{example}

  �N	N�~zz��:N�O��Q�

  \[

    \begin{bmatrix}

      1 & 3 & 5 \\

      2 & 4 & 6 \\

      3 & 6 & 9 \\

    \end{bmatrix}

    \begin{bmatrix}

      x_1 \\

      x_2 \\

      x_3 \\

    \end{bmatrix}

    =

    \begin{bmatrix}

      0 \\

      0 \\

      0 \\

    \end{bmatrix}

  \]

  ُ̑`O�S�N�Q�Q�[�^�v�ez�~.

\end{example}



\begin{solution}

  �V:N $\det{A} = 0$�Ee $A$ 
N�n�y�_N1\/f $A$ �[E�
N/f*N	N�~zz��̑�vN�R. �[E�
N�1u�N

  \[

    A \to

    \begin{bmatrix}

      1 & 0 & -1 \\

      0 & 1 & 2  \\

      0 & 0 & 0  \\

    \end{bmatrix}

    , r(A) = 2 < 3

  \]

  Ee�N $A$ �vRTϑ _b�vzz��/f*N	N�~zz��̑�v�N�~s^b��ُ�eX[(W^����. ��HN

  \[

    \begin{bmatrix}

      1 & 0 & -1 \\

      0 & 1 & 2  \\

      0 & 0 & 0  \\

    \end{bmatrix}

    \begin{bmatrix}

      x_1 \\

      x_2 \\

      x_3 \\

    \end{bmatrix}

    =

    \begin{bmatrix}

      0 \\

      0 \\

      0 \\

    \end{bmatrix}

  \]

  _N1\/f

  \[

    \left\{

    \begin{aligned}

      x_1 & = x_3   \\

      x_2 & = -2x_3

    \end{aligned}

    \right.

  \]

  ُ̑�1u�v/f $x_3$��f $\vec{x}$ 	gN*N�1u�^�_N1\/f $\vec{x}$ (WNag�v�~�N�~zz��	�
N. �[E�
N���|�v�~�^1\/f�STϑzz���v�~�^�Q�S�w5��v�y�sS $n - r(A)$. �b�N@b��W@x��|1\/f�S��~bُ*Nzz���v�W�^sS�S��V:Nُ̑/fN�~��N�S^��� $x_3$ ���S�N�_0R�W�^.



  ُ̑�S $x_3 = 1$�Ee�W@x��|:N

  \[

    \vec{\xi_1} =

    \begin{bmatrix}

      1  \\

      -2 \\

      1  \\

    \end{bmatrix}

  \]

\end{solution}



\begin{example}

  �N�V�~zz��:N�O��Q�

  \[

    \begin{bmatrix}

      1  & 2  & 3  & 4  \\

      5  & 6  & 7  & 8  \\

      9  & 10 & 11 & 12 \\

      13 & 14 & 15 & 16 \\

    \end{bmatrix}

    \begin{bmatrix}

      x_1 \\

      x_2 \\

      x_3 \\

      x_4 \\

    \end{bmatrix}

    =

    \begin{bmatrix}

      0 \\

      0 \\

      0 \\

      0 \\

    \end{bmatrix}

  \]

\end{example}



\begin{solution}

  �V:N

  \[

    A \to

    \begin{bmatrix}

      1 & 0 & -1 & -2 \\

      0 & 1 & 2  & 3  \\

      0 & 0 & 0  & 0  \\

      0 & 0 & 0  & 0  \\

    \end{bmatrix}

    , r(A) = 2

  \]

  �N�V�~�S�N�~�X[(W^�����_N1\/f

  \[

    \left\{

    \begin{aligned}

      x_1 & = x_3 + 2x_4   \\

      x_2 & = -2x_3 - 3x_4

    \end{aligned}

    \right.

  \]

  ُ̑�1u�v/f $x_3$ �T $x_4$��f $\vec{x}$ 	g$N*N�1u�^�_N1\/f $\vec{x}$ (WN*N�N�~s^b�
N.



  :N�NBlُ*Nzz���v�W�^�b�N���O���~'`�esQ���HNg�e�O�v1\/f�N

  \[

    x_3 = 1, x_4 = 0

  \]

  �Q�N

  \[

    x_3 = 0, x_4 = 1

  \]

  �S�_

  \[

    \vec{\xi_1} =

    \begin{bmatrix}

      1  \\

      -2 \\

      1  \\

      0  \\

    \end{bmatrix}

    , \vec{\xi_2} =

    \begin{bmatrix}

      2  \\

      -3 \\

      0  \\

      1  \\

    \end{bmatrix}

  \]

\end{solution}



\section{yr�_Tϑ}



% TODO: yr�_Tϑ



\section{L�R_}



\subsection{L�R_�v�QUOaIN}



�NpeaINw�v�_�la`��NHN�rN\��^pe�n܏�p.



	gsQ�QUOaIN�S�� \href{https://b23.tv/I25Kyqf}{�~'`�Npe�v,g(�}�b�N�Ndk�f�R_P[.



\begin{example}

  \[

    \det{AB} = \det{A} \cdot \det{B}

  \]



  �w5� $AB = ABE$�sS�b�w5� $E$ ۏL�$N!k�~'`�Sbc��~Ǐ$N!kR+R�v�~'`�Sbc $B$0$A$ T�vzz���TN!kteSO�v�~'`�Sbc $AB$ T�vzz���v)>e�k�O�6q�vI{�Npe<P
N�vXN.

\end{example}



\subsection{�ba�L�R_}



�NN��f�Reg� \href{http://www.ee.ic.ac.uk/hp/staff/dmb/matrix/proof003.html}{dk���c}.



\subsubsection{RWW�w5��vL�R_} \label{subsubsec:block-matrix-determinant}



RWW�w5��vL�R_�NN_:Nw��p�



$$

  \det{

    \begin{bmatrix}

      A & O \\

      C & D \\

    \end{bmatrix}

  } = \det{A} \cdot \det{D}

$$



\begin{proof}

  9hnc^INRI{�Sbc�vsQ�wƋ��SRWW�w5�sS

  \[

    \begin{bmatrix}

      A & O \\

      O & E \\

    \end{bmatrix}

    \begin{bmatrix}

      E & O \\

      C & E \\

    \end{bmatrix}

    \begin{bmatrix}

      E & O \\

      O & D \\

    \end{bmatrix}

  \]

  Neu.

\end{proof}



�[E�
N^INRI{�Sbc�w5���Bl�w5��S��FȎُ_P[/fb�z�v�baɉ�v�c�y:N�Sbc�w5�1\L��N��NT
N�Qyr�k�f.



�cNeg�f�NN_P[�



$$

  \det{

    \begin{bmatrix}

      A & B \\

      C & D \\

    \end{bmatrix}

  } = |A| \cdot |D - CA^{-1}B|

  = |D| \cdot |A - BD^{-1}C|

$$



9hnc^INRI{�Sbc�vsQ�wƋ��SRWW�w5�sS



$$

  \begin{bmatrix}

    E       & O \\

    CA^{-1} & E \\

  \end{bmatrix}

  \begin{bmatrix}

    A & B            \\

    O & D - CA^{-1}B \\

  \end{bmatrix}

$$



,{N*NI{�Sb�z



Tt؏�S�Nh�:y:N



$$

  \begin{bmatrix}

    E & BD^{-1} \\

    O & E       \\

  \end{bmatrix}

  \begin{bmatrix}

    A - BD^{-1}C & O \\

    C            & D \\

  \end{bmatrix}

$$



,{�N*NI{�Sb�z



d�dkKNY؏	g�YNh�:y�e�l



$$

  \begin{bmatrix}

    A & B - AC^{-1}D \\

    C & O            \\

  \end{bmatrix}

  \begin{bmatrix}

    E & C^{-1}D \\

    O & E       \\

  \end{bmatrix}

$$



�N�S



$$

  \begin{bmatrix}

    O            & B \\

    C - DB^{-1}A & D \\

  \end{bmatrix}

  \begin{bmatrix}

    E       & O \\

    B^{-1}A & E \\

  \end{bmatrix}

$$



���la�v/fُ̑	g



$$

  \det{

    \begin{bmatrix}

      M              & N_{u \times u} \\

      R_{v \times v} & O              \\

    \end{bmatrix}

  } = (-1)^{uv} \det{N} \cdot \det{R}

$$



sQ�N���_�e�l��la0RMR$N*N_P[�w5��Q�sz��^:N 	��[0�[҉0z��e��0-N��N_�Y�v� 



T$N*N_P[:N $-1$ �v�g!k�e0	��[0�[҉0��e��0-N��N_�Y�v� sS�S



\subsubsection{�w5�Џ�{�vL�R_}



$$

  |X_{m \times m} + A_{m \times n} B_{n \times m}| = |X_{m \times m}| |E_{n \times n} + B_{n \times m} X_{m \times m}^{-1} A_{m \times n}|

$$



\begin{proof}

  1u \cref{subsubsec:block-matrix-determinant} �_

  \[

    \text{�S_} = \det{

      \begin{bmatrix}

        X_{m \times m}  & A_{m \times n} \\

        -B_{n \times m} & E_{n \times n} \\

      \end{bmatrix}

    } = \text{�]_}

  \]

  1udk��

  \[

    M_{m \times m} = N_{m \times m} + k \begin{bmatrix}

      1      \\

      \vdots \\

      1      \\

    \end{bmatrix}

    \begin{bmatrix}

      1 & \cdots & 1

    \end{bmatrix}

  \]

  R

  \[

    \det{M} = \det{N} + k \sum_{i = 1}^{m} \sum_{j = 1}^{m} N_{ij}

  \]

  vQ-N $N_{ij}$ h�:y $N$ (W $i$ L� $j$ R�v�NpeYOP[_.

\end{proof}



\section{�v<O�[҉S}



\subsection{\texorpdfstring{:N�NHN	g $n$ *N�~'`�esQyr�_TϑN�[���v<O�[҉S�}{:N�NHN	g n *N�~'`�esQyr�_TϑN�[���v<O�[҉S�}}



��

\[

  \vec{\xi_1}, \cdots, \vec{\xi_n}

\] /f $n$ 6��w5� $A$ �v $n$ *N�esQyr�_Tϑ�T��[�^yr�_<P

\[

  \lambda_1, \cdots, \lambda_n

\]

R	g

\[

  A\vec{\xi_1} = \lambda_1 \vec{\xi_1}, \cdots, A\vec{\xi_n} = \lambda_n \vec{\xi_n}

\]

��

\[

  P = \begin{bmatrix}

    \vec{\xi_1} & \cdots & \vec{\xi_n}

  \end{bmatrix}

\]

Ee�_ $AP = P\varLambda$�vQ-N

\[

  \varLambda = \mathrm{diag}(\lambda_1, \cdots, \lambda_n)

\]

�S�V:N $P$ 1u $n$ *N�~'`�esQ�vTϑ�~b�Ee $P$ �n�y�S��Ee $P^{-1}AP = \varLambda$��_��.



\subsection{:N�NHN�[�[�y�w5�N�[���v<O�[҉S�}



�l	gُHN
YBg��[҉S�e^�1\/fM�bs^�e�@b�N�1\/f��[�N!k�WN�[��M�b�[hQs^�e.



ؚ-Nu���wS�`HNZP�g�{US�v1\/f(u*��_�lN*N*NM�1\/f.



�S�� \href{https://www.zhihu.com/question/38801697/answer/3157831805}{dk���c} �T \href{https://www.zhihu.com/question/38801697/answer/2584722969}{dk���c}.



\subsection{:N�NHN/fck�N�Sbc�}



�Na $n$ CQ�[�N!k�W $f(\vec{x}) = \vec{x}^T A \vec{x}$�vQ-N $A$ :N�[�[�y�w5����X[(Wck�N�Sbc $\vec{x} = Q\vec{y}$ \vQS:Nh�Q�W�

\[

  f(\vec{x}) = \vec{x}^T A \vec{x} = (Q\vec{y})^T A Q\vec{y} = \vec{y}^T Q^T A Q\vec{y} = \vec{y}^T \varLambda \vec{y} = \sum_{k = 1}^{n} \lambda_{k} y_k^2

\]

vQ-N $\varLambda = Q^T A Q = \mathrm{diag}(\lambda_1, \cdots, \lambda_n)$.



vQ��l�8^/fُ7h�v�



\begin{enumerate}

  \item Bl�Q $n$ *Nyr�_<P $\lambda_1, \cdots, \lambda_n$�

  \item Bl�Q $n$ *N�~'`�esQ�vyr�_Tϑ $\vec{\xi_1}, \cdots, \vec{\xi_n}$�

  \item \ $n$ *Nyr�_Tϑck�NS0USMOS:N $\vec{e_1}, \cdots, \vec{e_n}$�

  \item \ $\vec{e_1}, \cdots, \vec{e_n}$ 	cR�c�^�gb�w5� $Q$�

  \item �Q�Q�~Ǐck�N�SbcT�vh�Q�Wb�[�^�v�[҉�w5� $\varLambda = \mathrm{diag}(\lambda_1, \cdots, \lambda_n)$.

\end{enumerate}



vQ�[g͑���vN�p1\/f�~b0R�g�y�w5� $Q$ OvQ�n�� $Q^T A Q = \varLambda$.



‰�[ُ*N_P[�b�NO�S�sُ�T�w5��v<O/f�_{|<O�v.



�[E�
N�[�N�[�[�y�w5���_6qX[(W�S��w5� $P$ 	g $P^{-1} A P = \varLambda$    	g�v��O��BlBl�Q���sS@b	g�S���v $P$.



�ُ7h�v�w5� $P$ T�e_N�n�� $P^{-1} = P^{T}$���HNُ7h�v $P$ �[E�
N1\/fb�N�`~b�v�w5� $Q$    �S���N���n��ُ7h�v'`(�_N1\/fb�N�`���v $Q$ �N�FOُ�yZP�lǏ�NA~t.



wQ	gُ7h'`(��v�w5�_N���y:Nck�N�w5��	gsQck�N�w5��v'`(�(Wdk
N�QX���.



ُNb�N�Vegw
Nb��vek���MR$Nek�_}Yt��(uck�N�Sbc�l\�N!k�WS:Nh�Q�W1\/f�[�[�y�w5��v�v<O�[҉S�v�^(uKNN.



,{	Nekck�NS0USMOS�[E�
N/fck�N�w5��v'`(���Bl�v    ck�N�w5��v@b	gR�L�	�TϑKN���v�Nck�NN:NUSMOTϑ.



	gsQ :N�NHN\ck�NS0USMOST�vyr�_Tϑ	cR�c�^�gb�v�w5� $Q$ �N6q���O��	g $Q^{-1} A Q = \varLambda$ (Wdk
NZP�meQ����.



,{�Vek�T,{�Nek1\/f�~�|�v㉘�ek���N.



\section{ck�[�w5�0JSck�[�w5�0��[�w5�0JS��[�w5�0
N�[�w5�}



	gsQ�N!k�W�v�QUOaIN�S�� \href{https://www.bilibili.com/video/BV12N4y1H7Rn}{�N!k�Wvz�z/f*NeU��N!k�W�v�QUOaIN}



�N!k�W $\vec{x}^T A \vec{x}$ �[E�
N1\/f\Tϑ $\vec{x}$ �~Ǐ $A$ �v�~'`�SbcT�QN��]�pXN�v�~�g��Tϑ�pXN�v�~�g�vck�h:y@w$NTϑ�v9Y҉�`�Q.



wQSOeg��:Nck9Y҉\�N $\dfrac{\pi}{2}$�:N $0$ $N��W�v�:N�9Y҉'Y�N $\dfrac{\pi}{2}$�(Wdkb�N�{USR+R�y	N�y�`�Q:NTT0�W�v0�ST.



�ُ7h�v�w5� $A$ ��O�_�STϑzz���v@b	g^���Tϑ�~Ǐ�~'`�SbcT�OcTT�R�y $A$ :Nck�[�w5����Oc
N�STR:NJSck�[�w5���[�NvQ�N�vck�[�`�Q_N�S{|�k�[IN.



�[�N�ُ�y�`�Q/f^IN�vck�[�w5���x��g�rIN�vck�[�w5��sS�[ $A$ �Q�R
N�N�[�[�y�vP�6R.



b�N(Wdk�W@x
NegwwN�Nck�[�w5��v�vsQ�~���



\begin{itemize}

  \item $A$ �vyr�_<PGW:Nck.

\end{itemize}



*N�Nɉ�_ُ/fg��SO�sTTُN�i�_�v�~���GP�[ $A$ �v	gyr�_<P:N $0$���HNvQ�[�^�vyr�_Tϑ(W�SbcKNT1\�T��]�W�v�GP�[	gyr�_<P:N����HN�SbcKNT�T��]�ST.



�Ǐُ*N�~�����c�[�vI{�N}T��	g�



\begin{enumerate}

  \item $A_{n \times n}$ �vck�`'`cpe:N $n$�

  \item $A$ N $E$ TT��V:N $A$ �S�N(u1�R㉚[tR��6qT�byr�_<P_9h�S^XۏUSMOTϑsS�S�b�� $A$ /fck�[�w5�1\asT�[��Sb�|pehQ:N $1$ �vĉ��W��ُ*NS�l���[�[�^N*N�~'`�Sbc�

  \item T6�z��^;NP[_:Nck.

\end{enumerate}



b�N�QegwwJSck�[�N!k�W�

\[

  f(x_1, x_2, x_3) = (x_1 + x_2)^2 + (x_2 + x_3)^2 + (x_3 - x_1)^2

\]



�[>f6q
N��:N��NX[(W $x_1 = x_3 = -x_2 \neq 0$ O $f = 0$���HN>f6qُ/f*NJSck�[�w5�.



T7h�v�JSck�[�w5��vyr�_<PGW^���(W�R
NKNMRck�[�w5�yr�_<PGW:Nck�v�~���RJSck�[�w5��_	gyr�_Tϑ:N $0$.



9hnc
N���~���T�[҉S
N9e�S�w5��y�v'`(��b�N���_�_�Q�Q勌N!k�W�[�^�vĉ��W:N $y_1^2 + y_2^2$.



�[�N��[�w5�0JS��[�w5�0
N�[�w5��v'`(�
N�QX����ُ̑�p�QN�p��[�w5��v$R�[�e�l�



GYpe6�z��^;NP[_:N��vPpe6�z��^;NP[_:Nck.


��\chapter{
Ype}



\section{
Ype�VRЏ�{�v�QUOaIN}



�R�l�T�Q�leu�b�N@w͑�N�~XN�l.



\begin{example}

  \[

    (2 + \mathrm{i}) \cdot \mathrm{i} = -1 + 2\mathrm{i}

  \]

  �N
Ys^b�
Nw�$2 + \mathrm{i}$ ��e���el��N $\dfrac{\pi}{2}$.

\end{example}



\begin{example}

  \[

    (\sqrt{3} + \mathrm{i})(1 + \sqrt{3}\mathrm{i}) = 4\mathrm{i}

  \]

  �N
Ys^b�
Nw�$N���҉�v�R�!j��vXN.

\end{example}



\begin{figure}[H]

  \small

  \centering

  \begin{tikzpicture}[scale=0.8]

    % �~6RPWht��SQ<h

    \draw[->] (-3, 0) -- (3, 0) node[right] {$\Re$};

    \draw[->] (0, -1) -- (0, 4) node[above] {$\Im$};

    % \draw[step=1, gray, thin] (-3, -1) grid (3, 4);

    \fill (0,0) circle (0.03) node[below left] {$O$};



    % �~6R
Ype

    \draw[->, blue] (0, 0) -- (2, 1) node[below right] {$2 + \mathrm{i}$};

    \draw[->, blue] (0, 0) -- (-1, 2) node[above left] {$-1 + 2\mathrm{i}$};



    \draw[->, red] (0, 0) -- (1.732051, 1) node[above] {$\sqrt{3} + \mathrm{i}$};

    \draw[->, red] (0, 0) -- (1, 1.732051) node[above] {$1 + \sqrt{3} \mathrm{i}$};

    \draw[->, red] (0, 0) -- (0, 4) node[below right] {$4\mathrm{i}$};

  \end{tikzpicture}

\end{figure}



\begin{proof}

  �[
Ys^b�
NN
Ype $z$ ��!j�:N $l$���҉:N $\theta$. R

  \[

    z = l \cos{\theta} + \mathrm{i} \cdot l \sin{\theta} = l(\cos{\theta} + \mathrm{i} \sin{\theta})

  \]

  �� $z_1$ �[�^ $l_1, \theta_1$�$z_2$ �[�^ $l_2, \theta_2$. R

  \[

    \begin{aligned}

      z_1 z_2

       & = l_1(\cos{\theta_1} + \mathrm{i} \sin{\theta_1}) \cdot l_2(\cos{\theta_2} + \mathrm{i} \sin{\theta_2})                                                          \\

       & = l_1l_2 \left[\cos{\theta_1} \cos{\theta_2} - \sin{\theta_1} \sin{\theta_2} + \mathrm{i} (\sin{\theta_1} \cos{\theta_2} + \cos{\theta_1} \sin{\theta_2})\right] \\

       & = l_1l_2 [\cos{(\theta_1 + \theta_2)} + \mathrm{i} \sin{(\theta_1 + \theta_2)}]

    \end{aligned}

  \]

\end{proof}



1u
N	g
Ype�vXN�!j��vXN���҉�v�R.



Tt	g
Ype�vd��!j��vd����҉�v�Q.



�[�N�g�NЏ�{��)�w(u�e.



\section{
Ype�vcpeЏ�{}



$$

  \prod_{k = 1}^{n} a_k + b_k \mathrm{i} = \prod_{k = 1}^{n} \sqrt{a_k^2 + b_k^2} \cdot \left[\cos{\left(\sum_{k = 1}^{n} \arctan{\dfrac{b_k}{a_k}} \right)} + \mathrm{i} \sin{\left(\sum_{k = 1}^{n} \arctan{\dfrac{b_k}{a_k}} \right)} \right]

$$



vQ-N $a_k, b_k \in \mathbb{R}, n \in \mathbb{N}^*$



yr�k�v $a_k = a, b_k = b$ �e



$$

  (a + b\mathrm{i})^n  = \left(\sqrt{a^2 + b^2} \right)^n \cdot \left[\cos{\left(n \arctan{\dfrac{b}{a}} \right)} + \mathrm{i} \sin{\left(n \arctan{\dfrac{b}{a}} \right)} \right]

$$



�fyr�k�v�$a = \cos{\theta}, b = \sin{\theta}$ �e



$$

  (\cos{\theta} + \mathrm{i} \sin{\theta})^n = \cos{n\theta} + \mathrm{i} \sin{n\theta}

$$



1u�Ny�_�[t	g



$$

  (\cos{\theta} + \mathrm{i} \sin{\theta})^n = \sum_{k = 0}^{n} \mathrm{C}_n^k \cos^{n - k}{\theta} (\mathrm{i} \sin{\theta})^k

$$



�by��[�^�[�Z�萗_ $n$ 
P҉lQ_



$$

  \cos{n\theta} = \sum_{k = 0}^{\lfloor n / 2 \rfloor} \mathrm{C}_{n}^{2k} (-1)^k \cos^{n - 2k}{\theta} \sin^{2k}{\theta}

$$



$$

  \sin{n\theta} = \sum_{k = 0}^{\lfloor n / 2 \rfloor} \mathrm{C}_{n}^{2k + 1} (-1)^k \cos^{n - 2k - 1}{\theta} \sin^{2k + 1}{\theta}

$$



vQ-N $\lfloor n / 2 \rfloor$ h�:yTN�Ste


��\chapter{�v�~NW}


��\chapter{W%��f�~}



\section{W%��f�~N���QpeBl�[} \label{sec:conic-implicit-derivative}



W%��f�~
NN�pR�~�ez    �� \ref{subsec:implicit-derivative}.



W%��f�~YN�p$NR�~R�pޏ�~ (�S�yR�p&_) b__T
N.



\section{���eW} \label{sec:monge-circle}



-iW$N�W�vR�~�N�ph���:N�[W $x^2 + y^2 = a^2 + b^2$



\begin{figure}[H]

  \small

  \centering

  \begin{tikzpicture}

    % -iW

    \draw (0,0) ellipse (2 and 1.7321);

    \fill (2,0) circle (0.03) node[above right] {$2$};

    \fill (0,1.7321) circle (0.03) node[above left] {$\sqrt{3}$};



    % ���eWJS�_

    \def\r{2.6458} % sqrt(7) H" 2.6458

    \draw[gray,dashed] (0,0) circle (\r);



    % �p P

    \def\angle{60}

    \pgfmathsetmacro{\x}{\r*cos(\angle)}

    \pgfmathsetmacro{\y}{\r*sin(\angle)}

    \coordinate (P) at ({\x},{\y});

    \fill (P) circle (0.03) node[above right] {$P$};



    % �p M �T N

    \coordinate (M) at (-0.7133,1.6182);

    \coordinate (N) at (1.9228,0.4767);

    \fill (M) circle (0.03) node[above left] {$M$};

    \fill (N) circle (0.03) node[above right] {$N$};



    % ;uR�~ (���R�~)

    \draw (-2.8,0.9283) -- (2.8,2.7796);

    \draw (1.1547,2.8) -- (2.8,-2.1769);



    % ;u�W�v&{�S (�v҉h��)

    \pic [draw, angle radius=1mm] {right angle={M--P--N}};



    % PWht�

    \draw[->] (-3,0)--(3,0) node[right] {$x$};

    \draw[->] (0,-3)--(0,3) node[above] {$y$};



    % �S�p

    \fill (0,0) circle (0.03) node[below left] {$O$};

  \end{tikzpicture}

  \caption{���eW} \label{fig:monge-circle}

\end{figure}



\section{&q�p	N҉b_} \label{sec:focus-triangle}



�[ $\dfrac{x^2}{a^2} + \dfrac{y^2}{b^2} = 1 (a > b > 0)$ 	g $S_{\triangle{PF_1F_2}} = b^2 \cdot \tan{\dfrac{P}{2}}$.



�[ $\dfrac{x^2}{a^2} - \dfrac{y^2}{b^2} = 1 (a > b > 0)$ 	g $S_{\triangle{PF_1F_2}} = b^2 \cdot \cot{\dfrac{P}{2}}$.



�bir�~�e$N&q�p, �e&q�p	N҉b_.



\section{��_} \label{sec:directrix}



�[ $\dfrac{x^2}{a^2} + \dfrac{y^2}{b^2} = 1, (a > b > 0)$ 	g��_� $\dfrac{2b^2}{a}$.



�[ $\dfrac{x^2}{a^2} - \dfrac{y^2}{b^2} = 1, (a > b > 0)$ 	g��_� $\dfrac{2b^2}{a}$.



�[ $y^2 = 2px, (p \neq 0)$ 	g��_� $|2p|$.



\section{&q�p&_} \label{sec:focus-chord}



�[�y�_�s:N $e$ �vW%��f�~, Ǐ&q�p�v&_ $AB$ N&q�p@b(Wt��N҉�:N $\theta$, $|AF| = \lambda|FB|$.



R $|e \cdot \cos{\theta}| = |\dfrac{\lambda - 1}{\lambda + 1}|$.



�~T	N҉�Qpe�T�e�s $k$ �S�Sb_:NvQ�Nb__.



\section{�v�~N-iW�f�~�N�pX[(W$R+R_} \label{sec:line-ellipse-discriminant}



�[ $\dfrac{x^2}{a^2} + \dfrac{y^2}{b^2} = 1$ N $Ax + By + C = 0, (A \cdot B \neq 0)$



\begin{equation}

  \begin{cases}

    A^2 a^2 + B^2 b^2 = C^2 \iff \text{�vR} \\

    A^2 a^2 + B^2 b^2 > C^2 \iff \text{�v�N} \\

    A^2 a^2 + B^2 b^2 < C^2 \iff \text{�v�y}

  \end{cases}

\end{equation}



�[ $\dfrac{x^2}{a^2} - \dfrac{y^2}{b^2} = 1$ N $Ax + By + C = 0, (A \cdot B \neq 0)$



\begin{equation}

  \begin{cases}

    A^2 a^2 - B^2 b^2 = C^2 \iff \text{�vR} \\

    A^2 a^2 - B^2 b^2 < C^2 \iff \text{�v�N} \\

    A^2 a^2 - B^2 b^2 > C^2 \iff \text{�v�y}

  \end{cases}

\end{equation}



\section{9�l��[tNW%��f�~�vIQf['`(�} \label{sec:fermat-conic-optics}



\begin{theorem}[9�l��[t]

  IQ�NN�p O��SN�p�v(u�e;`/fg�w (GWS�N(�-Nh��s:N�zg�w)

\end{theorem}



GP�[W%��f�~��:N\�b�, R



\begin{itemize}

  \item �NW�_�S�Q�vIQ�S\T;`�V0RW�_.

  \item �N-iWN&q�p�S�Q�vIQ�S\T0R�SN&q�p.

  \item �N�bir�~&q�p�S�Q�vIQ�S\T;`/f�W�v�NvQ�Q�~.

  \item �N�S�f�~&q�p�S�Q�vIQ�S\T@b(W�v�~Ǐ�SN&q�p.

\end{itemize}



�S�N)R(u�N
N'`(�Bl�g�Nݍ�y�T (�]) �vg<P



N)R(uR-N�wƋ\OR�~�T�l�~	g�S\҉I{�NeQ\҉



\section{qQ&q�p�v-iWN�S�f�~} \label{sec:concentric-ellipse-hyperbola}


��\chapter{peR}



\section{peR�v�i�_}



peR (Number sequence) /f1upeW[�~b�v�^R. b�N؞���[peR\O�YN�v�[IN�



\begin{definition}[peR]

  N*N\�6qpeƖ $\mathbb{N}_0$ b $\mathbb{N}^*$  f\0R
YpeƖ $\mathbb{C}$ �v�Qpe $a$�1\/fN*N \strong{�ewzpeR}. b�N�8^(u $a_n$ egh�:ypeR�v,{ $n$ y��(u $\{a_n\}$ h�:ypeR,g��. 傚[IN�W/fN*N	gwzƖT�Rb�N�yvQ:N \strong{	gwzpeR}.

\end{definition}



S_6q��Qpe $a$ _N�S�N\�[IN�W f\0R�[peƖ $\mathbb{R}$�tepeƖ $\mathbb{Z}$�	gtpeƖ $\mathbb{Q}$ I{I{
YpeƖ $\mathbb{C}$ �vP[Ɩ.



\section{I{�]peR}



I{�]peR (Arithmetic sequence) /fN�y8^���vpeR. b�N�[I{�]peR\O�YN�v�[IN�



�[�NpeR $\{a_n\}$��X[(W8^pe $d$ O�_�[�Na�v $n \in \mathbb{N}$ $a_{n+1} - a_n = d$�R�y $a_n$ :N \strong{I{�]peR}�$d$ :N \strong{lQ�]}.



>f6q�I{�]peR�v�y�lQ_:N $a_n = a_1 + (n - 1)d$.



\section{Y6�I{�]peR}



\section{
N�R�p�v�i�_N'`(�}



N,��[�Qpe $f(x)$�� $\exists x_0 \in \mathbb{R}$ O $f(x_0) = x_0$�R�y $x = x_0$ :N $f(x)$ �vN6�
N�R�p�T�e	g $f(f(x_0)) = x_0$�f�_N6�
N�R�p_N/f�N6�
N�R�p.



N,��[peR $\{x_n\}$ 	g��c_ $x_{n+1} = f(x_n)$�� $\exists x_0 \in \mathbb{R}$ O $f(x_0) = x_0$�R�y $x = x_0$ :N $\{x_n\}$ �v
N�R�p. f�_��N�gy� $x_k$ :N
N�R�pRTpeRR`�[
N�S.



��/fpeR $\{x_n\}$ -N�vy��S
N0R
N�R�p $x_0$�FO��Y�cя�NTeg�vy���eg���cя $x_0$�R�ypeR $\{x_n\}$ 6e[e:N $x_0$.



<P�_�la�v/f
N�R�p�S��:N
Ype�_N�S��
NX[(W.



\subsection{
N�R�p�v3z�['`}



\subsection{N6��~'`��cpeR}



sS $\{x_n\}$ 	g $x_{n+1} = f(x_n) = px_n + q$�Bl $\{x_n\}$ �y�lQ_



\begin{enumerate}

  \item $p = 1$ �e:NI{�]peR.

  \item $p \mathbb{N}eq 1$ �e㉹ez $x_0 = f(x_0)$ �_
N�R�p $x_0 = \dfrac{q}{1 - p}$ T�R $\{x_n - x_0\}$ :NI{�kpeR.

\end{enumerate}



\subsection{R_��cpeR}



sS $x_n$ 	g $x_{n + 1} = f(x_n) = \dfrac{ax_n + b}{cx_n + d}$�Bl $\{x_n\}$ �y�lQ_.



ُ̑HQ㉹ez $x_0 = f(x_0)$ �_ $cx_0^2 + (d - a)x_0 - b = 0$.



\begin{enumerate}

  \item S_�S	gN� $x_0$ �e�$\{\dfrac{1}{x_n - x_0}\}$ :NI{�]peR.

  \item S_	g�O� $\alpha, \beta$ �e��la
Ype�_N/f$N*N	��$\{\dfrac{x_n - \alpha}{x_n - \beta}\}$ :NI{�kpeR.

\end{enumerate}





\subsection{N,��`b_}



N,��`b_NHQ�Q�㉹ez $x_0 = f(x_0)$ �_0R
N�R�p $x_0$.



�Q(W $x_{n + 1} = f(x_n)$ $N��T�e�Q�S $x_0$.



sS $x_{n + 1} - x_0 = f(x_n) - x_0$.



ۏL��Npe�Sb_TN,����_0RI{�]peRbI{�kpeRb__.



\subsection{
N�R�p:N��0
Ypeb
NX[(W�e}



:N���e�^S_�Q�Ppeb�Xy�TۏL��V_R�.



:N
Ype�epeR�_/fhTgpeR.




NX[(W�e�Q�vQ�N�e�l.



\subsection{�N6��~'`��cpeR}



sS $\{x_n\}$ 	g $x_{n + 1} = F(x_n, x_{n - 1}) = px_n + qx_{n - 1}$�Bl $\{x_n\}$ �y�lQ_��]�w $x_1, x_2$.



�� $\exists a, b$ O $x_{n + 1} - ax_n = b(x_n - ax_{n - 1})$.



	g $a + b = p, ab = -q$�sS $a, b$ :N $x^2 - px - q = 0$ �v�.



㉗_ $a, b$ T��NeQ�SHQ�vI{�kpeR�㉗_

\[

  x_n = \dfrac{x_2 - bx_1}{a - b}a^{n - 1} + \dfrac{x_2 - ax_1}{b - a}b^{n - 1}

\]



\section{N*NBl�T_}



\begin{example}

  Bl $\sum_{i = 1}^{n} i \cdot i!$.

\end{example}



\begin{solution}

  �la0R

  \[

    (n + 1)! = (n + 1) \cdot n! = n \cdot n! + n!

  \]

  Ee

  \[

    \sum_{i = 1}^{n} i \cdot i! = \sum_{i = 1}^{n} [(i + 1)! - i!] = (n + 1)! - 1

  \]

\end{solution}



\section{�g ��Qpe��fpeR
NI{_}



\subsection{/}�R�O}



\begin{example}

  Bl�� $\sum_{i = 1}^{n} \dfrac{1}{i + 1} < \ln{(1 + n)}$.

\end{example}



\begin{proof}

  )R(u $a_{n + 1} = S_{n + 1} - S_n$ Ջ\ $\ln{(1 + n)}$ �Qb/}�Rb__:N

  \[

    \sum_{i = 1}^{n} \ln{\dfrac{i + 1}{i}}

  \]

  RՋ��

  \[

    \dfrac{1}{x + 1} < \ln{\dfrac{x + 1}{x}}

  \] �[ $\forall x \ge 1$ b�z. ��

  \begin{gather*}

    f(x) = \ln{\dfrac{x + 1}{x}} - \dfrac{1}{x + 1}, (x \ge 1) \\

    f'(x) = -\dfrac{1}{x(x + 1)^2} < 0

  \end{gather*}

  Ee

  \[

    f(x) > \lim_{x \to +\infty} f(x) = 0

  \]

  sS $\dfrac{1}{x + 1} < \ln{\dfrac{x + 1}{x}}$ �[ $\forall x \ge 1$ �_��.

\end{proof}



\subsection{/}XN�O}



Bl�� $\dfrac{2}{n(n + 2)} < \prod_{i = 2}^{n} \ln{i}$



��)R(u $a_{n + 1} = \dfrac{T_{n + 1}}{T_n}$ Ջ\ $\dfrac{2}{n(n + 2)}$ �Qb/}XNb__:N $\prod_{i = 2}^{n} \dfrac{i - 1}{i + 1}$



RՋ�� $\dfrac{x - 1}{x + 1} < \ln{x}$ �[ $\forall x \ge 2$ b�z



Neu��_���_��.



1u�O�S�_b�N�S�N�Ǐ�[~b/}�R (XN) _�NՋ��/}�R (XN) �[�^Ty��v'Y\��Ǐ�S<Pv^Bl�[�_��T�V�csS���_����MQ�N�[�g ��Qpe
N�w@b�c.



\section{peR�v�gP�}



% TODO: peR�v�gP�



\section{Stolz �[t}



�[�N�NN$N�y�`�Q



\begin{itemize}

  \item �[�[peR $\{a_n\}, \{b_n\}$ N $\{b_n\}$ %N<hUS���X�v^	g $\lim_{n \to +\infty} b_n = +\infty$.

  \item �[�[peR $\{a_n\}, \{b_n\}$ N $\{b_n\}$ %N<hUS���Q�v^	g $\lim_{n \to +\infty} a_n = \lim_{n \to +\infty} b_n = 0$.

\end{itemize}



	g� $\lim_{n \to +\infty} \dfrac{a_{n + 1} - a_n}{b_{n + 1} - b_n} = A$�R $\lim_{n \to +\infty} \dfrac{a_n}{b_n} = A$ b�z.



\section{8^pey��~pe}



��	gpeR $\{u_n\}$�R�y $\sum_{k = 1}^{\infty} u_k$ :N�ewz�~pe.



$S_n = \sum_{k = 1}^{n} u_k$�R�y $\{S_n\}$ :N $\sum_{k = 1}^{\infty} u_k$ �v�R�TpeR.



� $\lim_{n \to \infty} S_n = S$�R�y $\sum_{k = 1}^{\infty} u_k$ 6e[e0R $S$�$r_n = S - S_n$ :N�~pe $\sum_{k = 1}^{\infty} u_k$ �vYO�.



\section{�~pe�v'`(�}



\begin{itemize}

  \item $\sum_{k = 1}^{\infty} u_k$ 6e[e $\implies \lim_{k \to \infty} u_k = 0$.

  \item �~pe�S�c	gP�y�
Nq_�T�~pe[ece'`.

  \item �~pe�R�b�ST6e[e��S�~pe
NN�[6e[e��~pe�R�b�ST�Sce��S�~peN�[�Sce.

\end{itemize}



\subsection{�~pe�v6e[e'`}



{|<O \cref{tbl:improper-integral-sum-convergence}�b�N�S�N\�~pe�v6e[e'`;`�~:N�YNh�<h�



\begin{table}[H]

  \small

  \centering

  \caption{�~pe6e[e'`h�} \label{tbl:series-convergence}

  \begin{tabular}{ccc}

    $\sum_{k = 1}^{\infty} u_k$ & $\sum_{k = 1}^{\infty} v_k$ & $\sum_{k = 1}^{\infty} (u_k \pm v_k)$ \\

    \hline

    �~�[6e[e                        & �~�[6e[e                        & �~�[6e[e                                  \\

    �~�[6e[e                        & ag�N6e[e                        & ag�N6e[e                                  \\

    ag�N6e[e                        & ag�N6e[e                        & 6e[e                                    \\

    6e[e                          & �Sce                          & �Sce                                    \\

    �Sce                          & �Sce                          & 
Nnx�[                                   \\

  \end{tabular}

\end{table}





\section{�~pe�v[ece'`$R+R�l}



\subsection{�k��$R+R�l}



�[ $\sum_{k = 1}^{\infty} u_k$ �T $\sum_{k = 1}^{\infty} v_k$ 	g



\begin{itemize}

  \item $\sum_{k = 1}^{\infty} v_k$ �~�[6e[eN�[@b	g��Y'Y�v $k$ 	g $|u_k| \leqslant |v_k|$�R $\sum_{k = 1}^{\infty} u_k$ _N�~�[6e[e.

  \item $\sum_{k = 1}^{\infty} v_k$ 
N�~�[6e[eN�[@b	g��Y'Y�v $k$ 	g $|u_k| \geqslant |v_k|$�R $\sum_{k = 1}^{\infty} u_k$ _N
N�~�[6e[e.

\end{itemize}



�la�,{�N�y�`�Q $\sum_{k = 1}^{\infty} u_k$ �S��ag�N6e[e.



\subsection{�gP��k��$R+R�l}



�[ $\sum_{k = 1}^{\infty} u_k$ �T $\sum_{k = 1}^{\infty} v_k$ N�[ $\forall{k \in \mathbb{N}^*}$ 	g $u_k, v_k \geqslant 0$��� $I = \lim_{k \to \infty} \dfrac{u_n}{v_n}$.



\begin{itemize}

  \item $I$ 
NX[(W�e $\sum_{k = 1}^{\infty} v_k$ �SceR $\sum_{k = 1}^{\infty} u_k$ �Sce

  \item $I = 0$ �e $\sum_{k = 1}^{\infty} v_k$ 6e[eR $\sum_{k = 1}^{\infty} u_k$ 6e[e

  \item $I > 0$ �e$N�T[ece

\end{itemize}



\subsection{�k<P$R+R�l}



�[ $\sum_{k = 1}^{\infty} u_k$�� $\lim_{k \to \infty} |\dfrac{u_{k + 1}}{u_k}| = \rho$�R



\begin{itemize}

  \item $\rho < 1$ �e�~�[6e[e

  \item $\rho > 1$ �e�Sce

  \item $\rho = 1$ �e
Nnx�[

\end{itemize}



\begin{proof}

  $\rho < 1$ �e�X[(W $r \in (\rho, 1)$ �T $N \in \mathbb{N}^*$ O $k > N$ �e	g $|u_{k + 1}| < r|u_k|$�Ee�S�_�[ $i \in \mathbb{N}^*$ 	g

  \[

    |u_{k + i}| < r^i |u_k|

  \]

  �

  \[

    \sum_{i = 1}^{\infty} |u_{k + i}| < |u_k| \sum_{i = 1}^{\infty} r^i

  \]

  N�S_6e[e��_�]_6e[e���]_1u�S�~pe�S�c	gP�y��_0R�Ee�S�~pe6e[e.



  $\rho > 1$ �eTt�S�_�Sce�$\rho = 1$ �e>N�OsS�S.

\end{proof}



\subsection{�yR$R+R�l}



% TODO:



\section{�P̑�S�~pe}



^��x͑�p��Sw \href{https://www.bilibili.com/video/av1554776715}{23 t^��N�24 t^_N��N��Nt^�Q�N!k��P̑�S�~pe�v@b	g��l�S�St 30 R��d�[} sS�S.





\backmatter
\chapter{参考答案}

\begin{answer}[\ref{prob:odd-even-union}] \label{ans:odd-even-union}
  \begin{solution}
    有 $f: O \to E, f(x) = x + 1$ 为双射函数, 则 $O, E$ 同势.
  \end{solution}
\end{answer}

\begin{answer}[\ref{prob:real-number-interval}] \label{ans:real-number-interval}
  \begin{solution}
    有 $f: \mathbb{R} \to (0, 1), f(x) = \dfrac{x}{1 + |x|}$ 为双射函数, 则 $\mathbb{R}$ 与 $(0, 1)$ 同势.
  \end{solution}
\end{answer}

\chapter{后记}


\renewcommand*{\bibnumfmt}[1]{\textbf{[#1]}}  % 加粗编号
\bibliographystyle{gbt7714-numerical}
\bibliography{reference}
\addcontentsline{toc}{chapter}{\bibname}

\printindex

\end{document}
